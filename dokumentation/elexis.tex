% *******************************************************************************
% * Copyright (c) 2007 by Elexis
% * All rights reserved. This document and the accompanying materials
% * are made available under the terms of the Eclipse Public License v1.0
% * which accompanies this distribution, and is available at
% * http://www.eclipse.org/legal/epl-v10.html
% *
% * Contributors:
% *    G. Weirich, U. Berner
% *
% *  $Id: elexis.tex 2538 2007-06-20 14:57:29Z rgw_ch $
% *******************************************************************************
% !Mode:: "TeX:UTF-8" (encoding info for WinEdt)
%
% Dies ist das Zentraldokument für die Elexis-Dokumentation. Einzelne
% Kapitel und Unterkapitel können mit \include eingesetzt werden.

\documentclass[a4paper,BCOR8.25mm,twoside]{scrbook}
\usepackage{german}
\usepackage[utf8]{inputenc}
\usepackage{makeidx}
\makeindex
% Hier ein etwas skurriler Block, der dazu dient, die Unterschiede
% zwischen PDFLaTeX und latex auszubügeln
% Grafiken müssen als png oder gif (für PDFLaTeX) und als eps (für Latex)
% vorhanden sein. Die Endung kann man beim \includegraphics jeweils weglassen,
% das System nimmt je nach Renderer die geeignete Variante.

\newif\ifpdf
\ifx\pdfoutput\undefined
	\pdffalse              	%normales LaTeX wird ausgeführt
\else
	\pdfoutput=1
	\pdftrue               	%PDFLaTeX wird ausgeführt
\fi

\ifpdf
	\usepackage[pdftex]{graphicx}
	\DeclareGraphicsExtensions{.pdf,.jpg,.png}
\else
	\usepackage[dvips]{graphicx}
	\DeclareGraphicsExtensions{.eps}
\fi

\usepackage{floatflt}
\usepackage[]{hyperref}
\usepackage{color}
\title{Das Elexis-Handbuch}
\begin{document}

\maketitle
%
%weiche Trennung wie? -> Trennvorschlag: "- Trennzwang: \-
%Sonderzeichen, Befehlszeichen wie auskommentieren? -> \backslash schreibt \
%hyperref mit weiteren Optionen?
%farbige Schrift? -> \color{red}
%Modell-pgn wie umwandeln?  -> png und eps (z.B. mit GIMP umwandeln)
\tableofcontents
\part{Einführung}
\chapter{Vorbemerkung zum Handbuch}
Dieses Handbuch ist -- ebenso wie das Programm selbst -- in laufender Entwicklung
begriffen. Die vorliegende Version ist nur ein Ausschnitt aus dem
Entwicklungsprozess. Es kann nicht garantiert werden, dass das Handbuch immer
exakt auf dem gleichen Entwicklungsstand ist, wie die Software selbst (und
umgekehrt). Es ist möglich, dass einige Kapitel noch unfertig sind und abrupt
enden. Wir denken dennoch, dass dieses Handbuch eine nützliche Hilfe beim
Einstieg in Elexis sein kann, und veröffentlichen es deswegen in dieser Form,
wobei wir Neuerungen jeweils in unregelmässigen Abständen ohne spezielle
Ankündigung in die zum Download bereitgestellte Version übertragen.

Wenn Sie immer die neuesten Entwicklungen sehen möchten, empfehlen wir Ihnen,
die Handbuch-Quellen wie auf Seite \pageref{dokumentation} beschrieben zu
installieren.

\chapter{Kurzbeschreibung und Schnelleinstieg}
Elexis -- die elektronische Praxis -- ist
\begin{itemize}
	\item ein Projekt zur gemeinschaftlichen Entwicklung eines quelloffenen (OpenSource) Praxisprogramms,
speziell angepasst an Schweizer Verhältnisse.
	\item  ein universelles Praxisprogramm:  Das Programm ist für alle Medizinalberufe geeignet. Ob Arzt- oder
Zahnarztpraxis, Physio- Ergo- Logopädie, Psychotherapie oder Homöopathie: Elexis lässt sich benutzerspezifisch
flexibel anpassen und beliebig mit Modulen erweitern.
	\item Aus der Praxis für die Praxis: Im Gegensatz zu den meisten anderen Programmen mit ähnlichen Anspruch entstand und entsteht Elexis in der Praxis für die Praxis. Das Programm wird von Fachleuten aus dem Medizinbereich entwickelt und unter realen Praxisbedingungen getestet.

	\item lauffähig unter Windows, Linux und MacOS X\footnote{Bei MacOS X
	allerdings mit der Einschränkung, dass die Textverarbeitung mit einem separaten
	Programm erfolgen muss, da die Einbettung von OpenOffice hier noch nicht
	möglich ist.}. Auch
	in gemischten Umgebungen mit verschiedenen Computersystemen am selben Server.
	\item Quelloffen: Elexis ist kein proprietäres Programm, Sie sind somit nicht
	von einer Herstellerfirma abhängig. Interessenten haben jederzeit Zugriff auf
	den Quellcode und können bei Bedarf die Funktionalität des Programmes erweitern.
Jedermann mit genügend Computerkenntnissen kann das System verstehen und weiterentwickeln.
Die Lizenz, unter der Elexis vertrieben wird, erlaubt eine solche Weiterentwicklung durch Dritte ausdrücklich
und ohne spezielle Bedingungen oder Kosten.

	\item Alles in einem: Krankengeschichte, Lagerbewirtschaftung,
	Bestellwesen, Abrechnung, Debitorenkontrolle, Agenda und vieles mehr.
  	\item kontaktfreudig: Kann Daten anderer Programme (sofern von deren
 	Hersteller vorgesehen) importieren und kann eigene Daten in Standardformate
 	exportieren.
 	\item Unterstützt von der Arbeitsgruppe Informatik der Schweizerischen
Gesellschaft für Allgemeinmedizin %\href{URL}{Sgam.informatics}
\end{itemize}
\pagebreak[3]
\textbf{Unterstützung}
Elexis enthält alle Komponenten um per Doppelklick starten zu können.
Eine elektronische Krankengeschichte ist allerdings ein komplexes Gebilde, und nicht vom Stand weg zu
beherrschen. Wir versuchen Ihnen mit unserer Dokumentation den Einstieg zu
erleichtern. Wenn Sie sich lieber nicht mit Computerinterna beschäftigen, bieten
wir Ihnen auch Unterstützung in Form von Installations- und Wartungsverträgen
an. Gerne machen wir Ihnen auf Anfrage eine Offerte.

% *******************************************************************************
% * Copyright (c) 2007 by Elexis
% * All rights reserved. This document and the accompanying materials
% * are made available under the terms of the Eclipse Public License v1.0
% * which accompanies this distribution, and is available at
% * http://www.eclipse.org/legal/epl-v10.html
% *
% * Contributors:
% *    G. Weirich
% *
% *  $Id: elexis.tex 2453 2007-05-30 15:16:13Z rgw_ch $
% *******************************************************************************
% !Mode:: "TeX:UTF-8" (encoding info for WinEdt)

\section{Standard-Installation}
\label{easyistall}
\index{installation}
Um den raschen Einstieg zu erleichtern haben wir für Windows, Linux und Mac leicht installierbare und sofort lauffähige Pakete zusammengestellt, die auch schon eine Beispiel-Datenbank enthalten.
Vergewissern Sie sich bitte zunächst, ob Ihr System von Elexis unterstützt wird (Anhang \ref{systemvoraussetzungen} dieses Handbuchs, Seite \pageref{systemvoraussetzungen})
\subsection{Windows 2000/XP}
(NB: Vista gehört noch nicht zu den unterstützten Betriebssystemen)
\begin{itemize}
	\item Laden Sie \href{http://www.elexis.ch/download.php?file=elexis-windows}{www.elexis.ch/download.php?file=elexis-windows} herunter (ca. 150MB)
	\item starten Sie die heruntergeladene Datei elexis-windows-x.y.z.exe und folgen Sie den Anweisungen am Bildschirm.
    \item Programm starten: Doppelklick auf das Symbol \glqq Elexis\grqq{} auf Ihrem Desktop oder das Programm \glqq Elexis\grqq{} im Startmenü auswählen.
	\item Deinstallation: Klicken Sie auf \glqq Elexis deinstallieren\grqq im  Startmenü.
\end{itemize}

\subsection{Linux}
(getestet mit SuSE 10.0, Kubuntu 6.06, 6.10 und 7.04, sowie  Xubuntu 6.10 und
7.04.)
\begin{itemize}
	\item Laden Sie \href{http://www.elexis.ch/download.php?file=elexis-linux}{www.elexis.ch/download.php?file=elexis-linux} herunter (ca. 90MB)
	\item Entpacken Sie das	heruntergeladene Archiv	elexis-linux-x.y.z.tgz an eine beliebige Stelle\footnote{Z.B. mit dem Befehl tar -xzf elexis-linux-x.y.z.tgz}
    \item Programm starten: Starten Sie einfach das Programm \footnote{meistens so: ./elexis}. Selbstverständlich können Sie auch unter Linux einen Shortcut auf dem Desktop
    anlegen, das dazu nötige Vorgehen variiert aber je nach desktop environment.

 	\item Textprogramm einrichten: Da bei den empfohlenen Linux.Distributionen
 	OpenOffice ohnehin integriert ist, liefern wir es beim Linux-Installer nicht
 	mit. Daher ist deiser auch erheblich kleiner, als der Windows-Installer. Dafür
 	ist noch ein wenig 	\glqq Handarbeit\grqq{}notwendig, um die Zusammenarbeit
 	von Elexis mit OpenOffice einzurichten (Anleitung für Kubuntu/Xubuntu, SuSE
 	analog mit Yast statt mit apt-get):
	\begin{itemize}
	 	\item öffnen Sie eine Konsole und geben Sie ein: \textit{sudo apt-get install
	 	openoffice.org-officebean} 
		\item öffnen Sie in Elexis das Menü \textsc{Datei - Einstellungen} und suchen
		Sie dort die Seite \textsc{Textverarbeitung} auf. Markieren Sie den Punkt
		\glqq OpenOffice Wrapper\grqq{}.
		\item Gehen Sie dann im selben Dialog zur Seite \textsc{OPenOffice.org}.
		Suchen Sie mit dem Knopf \textsc{Durchsuchen} Ihr OpenOffice-
		Programmverzeichnis auf (in aller Regel wird dies bei Kubuntu
		/usr/lib/openoffice/program sein). Klicken Sie dann auf \textsc{Anwenden} und
		schliessen Sie den Dialog mit \textsc{OK}
		\item Wichtig: Verlassen Sie Elexis, warten Sie einige Sekunden und starten
		Sie neu.
    \end{itemize}
  \item Deinstallation: Löschen Sie den beim Entpacken erstellten Ordner. Das
 ist alles.
\end{itemize}

\subsection{Apple Macintosh OS-X}\
(Getestet Version 10.4 (Tiger). Einschränkung: Kein integriertes Textsystem)
\begin{itemize}
	\item Laden Sie \href{http://www.elexis.ch/download.php?file=elexis-macosx}{www.elexis.ch/download.php?file=elexis-macosx} herunter (ca. 25 MB)
	\item Entpacken Sie das heruntergeladene Archiv elexis-macosx-x.y.z.zip an eine beliebige Stelle.
    \item Programm starten: Öffnen Sie das heruntergeladene Verzeichnis und doppelklicken Sie auf das Programm \glqq Elexis\grqq{}
	\item Deinstallation: Löschen Sie den beim Entpacken erstellten Ordner. Das ist alles.
\end{itemize}

Bei allen Betriebssystemen wird beim ersten Start die Einrichtung des Programms komplettiert.
Geben Sie als Anwender und als Passwort jeweils \textbf{\textit{test}}\index{Passwort}\index{Anfangspasswort}
ein und beenden Sie das Programm gleich wieder. Warten Sie nach dem Programmende mindestens 30 Sekunden (Aufbau der Demo-Datenbank). Beim nächsten Start sollte alles normal funktionieren (ebenfalls wieder als \glqq test\grqq mit dem Passwort \glqq test\grqq anmelden).

Wir empfehlen, dass Sie dann zum \glqq warmwerden\grqq{} die geführte Tour (s.S.
\pageref{tour}) machen.

\bigskip
Nach der Installation haben Sie ein System mit einer vorkonfigurierten einfachen
lokalen Demo-Datenbank und einigen Beispiel-Patienten. Wenn Sie dieses System in ein
\glqq echtes\grqq{} Praxisprogramm umwandeln wollen, lesen Sie bitte in Anhang
\ref{vollversion} (S. \pageref{vollversion}, wie das geht.



%%%%%%%%%%%%%%%%%%%%%%%%%%%%%%%%%%%%%%%%%%%%%%%%%%%%%%%%%%%%%%%%%%%%%%%%%%%%%%
\part{Geführte Tour}
\chapter{Eine neue Patientin erfassen}
\label{tour}
	% *******************************************************************************
% * Copyright (c) 2007 by Elexis
% * All rights reserved. This document and the accompanying materials
% * are made available under the terms of the Eclipse Public License v1.0
% * which accompanies this distribution, and is available at
% * http://www.eclipse.org/legal/epl-v10.html
% *
% * Contributors:
% *    G. Weirich - initial implementation
% *
% *  $Id: patneu.tex 2645 2007-06-28 12:33:35Z rgw_ch $
% *******************************************************************************
% !Mode:: "TeX:UTF-8" (encoding info for WinEdt)

Starten Sie Elexis durch Doppelklick auf das Programmsymbol.
Nach einem kurzen Moment erscheint ein Fenster etwa wie in Abb. \ref{fig:startbild} (sofern sie die
Demo-Datenbank verwenden, andernfalls sind die Fenster natürlich
leer).\footnote{Die Abbildungen dieser Tour entstammen Windows XP. Unter anderen Betriebssystemen wird das Aussehen leicht abweichend sein.}
 \begin{figure}[ht]
	\includegraphics[width=0.8\textwidth]{images/einf0}
	\caption{Elexis Startbildschirm}
	\label{fig:startbild}
\end{figure}
\section{Patientendaten erfassen}
Aktivieren sie mit einem Klick die Ansicht 'Patienten' und schreiben Sie in die Eingabefelder Name und Vorname der neuen Patientin.
Falls die Patientin schon einmal erfasst worden ist erscheint ihr Name, in unserem Fall, wo keine Patientin dieses Namens vorhanden ist,
wird \glqq Keine Daten\grqq{}angezeigt (s. Abb. \ref{fig:patname}).
\begin{figure}[ht]
	\includegraphics{images/einf1}
	\caption{Patientennamen eingeben}
	\label{fig:patname}
\end{figure}
Klicken Sie dann auf den roten Stern oben rechts, um eine Patientin mit diesen
Daten neu anzulegen. Es erscheint ein Dialogfenster (Abb. \ref{fig:patdata}), wo
Sie die Angaben in die entsprechenden Felder eingeben können.
\begin{figure}[ht]
	\includegraphics{images/einf2}
	\caption{Patientendaten ergänzen}
	\label{fig:patdata}
\end{figure}
Sie brauchen die hier verlangten Daten nicht vollständig einzugeben, sondern einfach soweit Sie diese im Moment kennen.
Sie müssen also im Notfalldienst nicht zuerst die vollständigen Daten eingeben, bevor Sie mit der Behandlung beginnen können.
Erfassen Sie z.B. nur Name und Geburtsdatum und überlassen Sie den Rest Ihrer MPA. Für Elexis ist ein neuer Patient in dem Moment bekannt,
wo Sie 'OK' klicken -- egal wieviele Daten zu diesem Zeitpunkt eingegeben sind.

\section{Falldaten erfassen}
Bei einer neuen Patientin müssen Sie zunächst einen \glqq Fall\grqq{} erstellen, dem
die Konsultation zugeordnet werden kann. Ein Fall sammelt alle Konsultationen, die einem gemeinsamen Garanten
zugeordnet werden können. Klicken Sie also in der \glqq Fälle\grqq{}-Ansicht
(Abb. \ref{fig:faelle1}) auf den \glqq Neu\grqq{}-Stern.
\begin{figure}[ht]
	\includegraphics{images/einf3}
	\caption{Fälle-Ansicht}
	\label{fig:faelle1}
\end{figure}
Dadurch öffnet sich ein Dialog, in dem Sie wiederum die Angaben eintragen
können, soweit diese Ihnen bekannt sind (Abb. \ref{fig:falldetail})
\begin{figure}[ht]
	\includegraphics{images/einf4}
	\caption{Fall-Details}
	\label{fig:falldetail}
\end{figure}
Spätestens zur Rechnungsstellung müssen dann allerdings die notwendigen Angaben (Debitor, Kostenträger und Versicherungsnummer bzw. Fallnummer)
eingegeben werden.
Nach dem Klick auf OK haben Sie den neuen Fall erstellt. Bei weiteren
Konsultationen kann man sich diesen Schritt natürlich sparen.

Als nächstes erstellen wir eine neue Konsultation, wieder mit dem nun schon
bekannten Stern-Symbol (Abb. \ref{fig:neuekons}.
\begin{figure}[ht]
	\includegraphics{images/einf5}
	\caption{Neue Konsultation}
	\label{fig:neuekons}
\end{figure}
\pagebreak[2]
Danach können wir mit dem KG-Eintrag beginnen (Abb. \ref{fig:KG}).
\section{Krankengeschichte führen}
\begin{figure}[ht]
	\includegraphics[width=0.9\textwidth]{images/einf6}
	\caption{KG-Eintrag}
	\label{fig:KG}
\end{figure}
Der KG-Eintrag kann einfache Textformatierungen enthalten, Textbausteine können
beliebig definiert und über eine konfigurierbare shortcut-Taste aufgerufen werden. Die Verrechnung erfolgt dann entweder über ein Tastaturmakro oder per Maus.
Nach Fertigstellung des Eintrags (auch vor oder während des Eintragens) können
Sie durch Klicken auf \glqq Verrechnung\grqq{} die Leistungen-Ansicht öffnen
(Abb. \ref{fig:Verrechnung}. Analog können sie durch klicken auf
\glqq Behandlungsdiagnosen\grqq{} die Diagnosen-Ansicht öffnen.\bigskip
\begin{figure}[ht]
	\includegraphics[width=0.9\textwidth]{images/einf7}
	\caption{Verrechnungs-Fenster}
	\label{fig:Verrechnung}
\end{figure}
Dieses Fenster enthält alle im System vorgesehenen Leistungscode-Systeme, sowie eine Seite mit selbstdefinierten Leistungsblöcken.
Sie können entweder einen ganzen Block oder einzelne Leistungen aus dem Block oder aus einem anderen Leistungsfenster (Tarmed etc.)
ins \glqq Verrechnung\grqq{}-Feld ziehen (drag an drop).

Genau gleich lassen sich zur Konsultation auch Diagnosen zuordnen, auch hier hat man die Wahl zwischen allen im System integrierten
Diagnosecodesystemen (beliebig anzupassen und erweiterbar).
\newpage{3}
Zum Abschluss möchten wir Ihnen noch zeigen, wie Sie eine View zur besseren Übersicht auf volle Bildgrösse und wieder
in ihren Originalzustand bringen können: Sie brauchen dazu nur auf dem Karteireiter der entsprechenden View doppelt zu
klicken (Abb. \ref{fig:viewmax}).
\begin{figure}[ht]
	\includegraphics[width=0.9\textwidth]{images/einf8}
	\includegraphics[width=0.9\textwidth]{images/einf9}
	\caption{View maximieren}
	\label{fig:viewmax}
\end{figure}


\chapter{Benutzeroberfläche einrichten}
	\label{customize}
	% *******************************************************************************
% * Copyright (c) 2007 by Elexis
% * All rights reserved. This document and the accompanying materials
% * are made available under the terms of the Eclipse Public License v1.0
% * which accompanies this distribution, and is available at
% * http://www.eclipse.org/legal/epl-v10.html
% *
% * Contributors:
% *    G. Weirich - initial implementation
% *
% *  $Id: customize.tex 2773 2007-07-10 15:58:33Z rgw_ch $
% *******************************************************************************
% !Mode:: "TeX:UTF-8" (encoding info for WinEdt)

\section{Funktionsprinzip}
Hervorstechendstes Merkmal vom Elexis ist die grosse Flexibilität. Wenn Sie ein anderes Praxisprogramm gewöhnt sind,
wird Ihnen die Bedienung von Elexis vielleicht etwas ungewöhnlich vorkommen. Wir möchten deshalb hier zunächst einige
grundsätzliche Konzepte erläutern.
\index{Bedienungskonzepte}
 \subsection{Schreibtisch / Perspektive}
Stellen Sie sich Ihren Arbeitstisch vor. Vermutlich werden Sie sich im Lauf der Zeit angewöhnt haben,
bestimmte Dinge an einen bestimmten Ort auf Ihrem Schreibtisch zu legen, also
Arbeitsfunktionen einem Ort zuzuordnen, wo sie sie jeweils
(idealerweise) leicht wiederfinden. Ihre Anordnung ist nicht unbedingt dieselbe wie bei jemand anderem,
der dasselbe Schreibtischmodell besitzt.

Das Programmfenster von Elexis ist so ein Schreibtisch (s. fig. \ref{fig:tour1}.Es ist in
keiner Weise festgelegt, welche Funktion
wo zu finden ist, ja es ist nicht einmal festgelegt, welche Elemente überhaupt
auf dem Schreibtisch erscheinen, und welche vielleicht irgendwo
in einer Schublade verstaut sind und nur bei Bedarf hervorgeholt werden müssen.

%\usepackage{graphics} is needed for \includegraphics
\begin{figure}[htp]
\begin{center}
  \includegraphics[width=0.9\textwidth]{images/tour1}
  \caption{Standard-Perspektive}
  \label{fig:tour1}
\end{center}
\end{figure}

Eine Anordnung von Arbeitsflächen nennen wir eine 'Perspektive' (Perspective). Die einzelnen Funktionseinheiten
(oben 'Patienten' und 'Patienten-Detail'), aus denen sich die Perspektive zusammensetzt bezeichnet
man als 'Ansicht' (View).

\subsubsection{Perspektive und Views}

Oben (fig. \ref{fig:tour1} sehen sie als Beispiel eine Perspektive, die für
einen kleinen Bildschirm geeignet ist, es zeigt einen
Screenshot auf einem 15-Zoll-TFT-Monitor. Die Ansichten \glqq
Patienten\grqq{}(links) und \glqq Patient Detail\grqq{}(rechts) liegen obenauf,
andere Ansichten sind dahinter angeordnet, so dass nur ein Karteireiter oben
zu sehen ist.
Auf einem grösseren Bildschirm würden Sie vermutlich eine andere Anordnung
bevorzugen: fig. \ref{fig:tour2} zeigt einen Screenshot auf einem
17-Zoll-TFT-Monitor mit mehreren Views gleichzeitig.

%\usepackage{graphics} is needed for \includegraphics
\begin{figure}[htp]
\begin{center}
  \includegraphics[width=0.9\textwidth]{images/tour2}
  \caption{Komplexere Perspektive}
  \label{fig:tour2}
\end{center}
\end{figure}
\textbf{Ansichten}

\subsubsection{Ansichten / Views}
Jede Ansicht entspricht einer bestimmten Funktionalität. Im abgebildeten Fenster sehen sie die Ansicht einer
Patientenliste (links) und die Details des 'aktiven' Patienten (rechts). Es gibt weitere Ansichten wie KG-Eintrag des aktuellen Patienten, eine Liste aller KG-Einträge, die Fixmedikation, die Rezepte,die Arbeitsunfähigkeitszeugnisse,
die Agenda etc. Jede Ansicht ist eine definierte \glqq Sicht\grqq auf die vorhandenen Daten,
daher der Name \glqq Ansicht\grqq. Sie lassen sich über die Reiter aktivieren. Die
Reiter selber lassen sich beliebig anordnen, aktivieren oder deaktivieren.

Egal wie Sie die Views angeordnet haben, jede View lässt sich zur besseren
Übersicht jederzeit auf Vollbildgrösse bringen, indem man auf den Reiter
doppelklickt (s. Abb. \ref{fig:tour3}).

\begin{figure}[htp]
\begin{center}
  \includegraphics[width=0.9\textwidth]{images/tour3}
  \caption{View maximieren}
  \label{fig:tour3}
\end{center}
\end{figure}

\subsubsection{Views und Perspektiven anpassen}


In der Standard-Startperspektive ist links eine \glqq Startleiste\grqq{} zu
sehen. Diese führt Sie zu vordefinierten Perspektiven - es erscheinen in diesen
jeweils die passenden Ansichten. Die Werkzeugleiste führt, wie bei anderen
Programmen üblich, zu verschiedenen Funktionen. Jede Ansicht hat einen
Karteireiter, über den sie in den Vordergrund gebracht oder maximiert werden kann.

Sie können die Fensterinhalte und Ansichten völlig frei gestalten. 
%Die nächsten Seiten werden Ihnen zeigen,
%wie einfach Sie Änderungen vornehmen und die Software so Ihren Arbeitsabläufen anpassen können.
%Änderungen der Perspektiven und Ansichten  ?ub: Doppel der subsection ?

Die Programmfenster-Inhalte lassen sich jederzeit anpassen:
\begin{itemize}
  \item Sie können nicht benötigte Ansichten entfernen damit sie mehr Platz für
  die verbleibenden Ansichten haben
	\item können Views in der Horizontalen und Vertikalen vergrössern oder verkleinern
	\item können Views an beliebige andere Stellen des Bildschirms schieben (indem Sie sie
	an den Reitern mit gedrückter linker Maustaste \glqq festhalten\grqq{})
\end{itemize}

Jede Zusammenstellung kann als Perspektive gespeichert werden - und ist als
solche auf einfache Art wieder aufrufbar. 
%Die folgenden Beispiele zeigen Ihnen das Vorgehen.

\subsection{Perspektiven einrichten und speichern}
Sie können nicht nur eine Perspektive erstellen, sondern beliebig viele. Ihre MPA braucht möglicherweise eine
andere Perspektive als Sie selber, z.B. wünscht sie sich die Agenda gross. Oder Sie selber verwenden unterschiedliche
Perspektiven, z.B. eine für Konsultationen und eine andere für die Buchhaltung oder wenn
Sie einen Bericht schreiben. Perspektiven lassen sich mit Elexis in wenigen Schritten zusammenstellen:
\begin{itemize}
 	\item Benötigte Ansichten öffnen
	\item Die Ansichten an die gewünschte Stelle schieben und auf die gewünschte Grösse bringen
	\item Die Ansichten an die gewünschte Stelle schieben und auf die gewünschte Grösse bringen
	\item Perspektive speichern

\end{itemize}


%%%%%%%%%%%%%%%%%%%%%%%%%%%%%%%%%%%%%%%%%%%%%%%%%%%%%%%%%%%%%%%%%%%%%%%%%%%%%%%
\part{Systematische Referenz}
\chapter{Konzepte}
	% *******************************************************************************
% * Copyright (c) 2007 by Elexis
% * All rights reserved. This document and the accompanying materials
% * are made available under the terms of the Eclipse Public License v1.0
% * which accompanies this distribution, and is available at
% * http://www.eclipse.org/legal/epl-v10.html
% *
% * Contributors:
% *    G. Weirich
% *
% *  $Id: konzepte.tex 2932 2007-07-29 06:21:06Z rgw_ch $
% *******************************************************************************
% !Mode:: "TeX:UTF-8" (encoding info for WinEdt)

\section{Kontakte}
\label{kontakt}
In Elexis ist jede Person oder Firma, die in irgendeiner Beziehung zur Praxis
steht, zunächst mal ein \glqq Kontakt\grqq{}. Kontakte werden in der
Kontakt-Perspektive eingegeben oder geändert.
\begin{flushleft}
    \includegraphics{images/contactperspective}
\end{flushleft}


Es gibt folgende Typen von Kontakten:
\begin{itemize}
  \item Person
	\begin{itemize}
  		\item Mandant
  		\item Anwender
  		\item Patient
  		\item Andere
    \end{itemize}
    \item{Organisation}
    \begin{itemize}
      \item{Labor}
      \item {Andere}
    \end{itemize}
\end{itemize}


\section{Anwender und Mandanten}
Jemand, der eine Rechnungsstelle (in der Schweiz z.B. eine eigene ZSR-Nummer) hat, ist ein \textit{Mandant}. Jeder Vorgang in Elexis (Konsultation, Labor, Rezept etc.) läuft immer unter Verantwortung und auf Rechnung genau eines Mandanten. \index{Mandant}

\medskip

Jemand, der das Programm bedienen darf, ist ein \textit{Anwender}. Ein Anwender arbeitet immer im Auftrag eines bestimmten Mandanten.

Zu jedem Zeitpunkt gibt es in Elexis also einen aktuellen Mandanten und einen aktuellen Anwender.
\index{Anwender}Mandant und Anwender können auch identisch sein (Wenn der
Mandant selbst am PC arbeitet).
Ein Anwender kann auch die Mandantenzuordnung ändern (Wenn eine MPA in einer Gruppenpraxis beispielsweise für unterschiedliche
Mandantinnen arbeitet).
\index{Gruppen}\index{Rechte}
Anwender haben bestimmte, individuell einstellbare Rechte, mit denen man sehr
fein steuern kann, wer welche Aktionen innerhalb von Elexis steuern kann.
Anwender können auch in Gruppen zusammengefasst sein, die bestimmte gemeinsame
Rechte definieren (z.B. Gruppen \glqq MPAs\grqq{} oder \glqq Ärzte\grqq{}). Eine
spezielle Gruppe ist die Gruppe \glqq Admin\grqq{}: Wer zu dieser Gruppe gehört,
hat automatisch \textit{alle} Rechte.
\textbf{Wichtig}: Auch wenn Ihnen das zunächst unlogisch erscheinen mag: Auch
der Chef sollte normalerweise nicht als Admin \index{Administrator} arbeiten.
Der Grund ist, dass der Admin-Account auch irreversible Löschungen und andere
sehr unangenehme Veränderungen erlaubt. Wie schnell hat man in der Hektik des
Alltags mal einen falschen Knopf geklickt!
Deswegen: Arbeiten Sie im Alltag mit einem Account, der genau diejenigen Rechte
hat, die Sie auch im Alltag brauchen. Erstellen Sie für sich einen zweiten
Account, welcher der Gruppe Admin zugeordnet ist, und melden Sie sich nur dann
unter diesem Account an, wenn es wirklich notwendig ist.

Das Konzept der Gruppen und Rechte ist ab Seite \pageref{sec:gruppen} ff.
genauer erklärt.

\section{Konsultationen, Fälle, Garanten und Kostenträger}
\index{Konsultation}\index{Fall}\index{Garant}\index{Krankenkasse}\index{Kostenträger}
\index{Leistungserbringer}\index{Abrechnung}Jeder in Elexis festgehaltene Kontakt zwischen Praxispersonal und Patient ist eine \textit{Konsultation}. Wenn die Konsultation verrechnet wird, dann geht die Verrechnung zugunsten desjenigen Mandanten, für welchen der eingeloggte Anwender tätig war.

Jede Konsultation ist auch einem \textit{Fall} zugeordnet. Ein Fall ist hier eher eine versicherungstechnische, als eine medizinische Einheit: Der Fall sammelt alle Konsultationen, welche mit demselben Abrechnungssystem (s. \ref{settings:abrechnungssystem} auf S. \pageref{settings:abrechnungssystem}) abgerechnet werden. Dies kann manchmal identisch mit dem medizinischen Fallbegriff sein (Ein Unfall, welcher über einen bestimmten Versicherer mit einer bestimmten Fallnummer abgerechnet wird), oder er kann auch keinen Zusammenhang mit einem medizinischen Fall haben (z.B. wird in der Schweiz im Allgemeinen ein allgemeiner Fall \glqq Krankheit\grqq{} erstellt werden, der alle KVG-Konsultationen sammelt.

%Jeder Fall hat einen \textit{Garanten}, das ist die Person oder Organisation (=Kontakt), welche die Rechnung erhält und primärer Schuldner ist.

%Ausserdem  hat ein Fall einen \textit{Kostenträger}, das ist die Person oder Organisation (=Kontakt), welche die Rechnung letztlich bezahlt.

%Garant und Kostenträger können selbstverständlich identisch sein, beispielsweise bei Privatrechnung oder bei Behandlungen, die im Tiers Payant-System abgerechnet werden (in der Schweiz z.B. UVG und IV). Beim Tiers Garant System dagegen (KVG in den meisten Kantonen), ist der Patient selbst Garant, die Krankenkasse aber Kostenträger.

%Ein Fall kann immer nur einen Garanten, einen Kostenträger und einen Patienten enthalten, kann aber durchaus  Konsultationen mehrerer Mandanten beinhalten. (Es wird dann für jeden Mandanten eine separate Rechnung erstellt).

Ein Fall kann immer nur einen Patienten und ein Abrechnungssystem haben, kann aber durchaus  Konsultationen mehrerer Mandanten beinhalten. (Es wird dann für jeden Mandanten eine separate Rechnung erstellt).

\section{Artikel und Lager}
\index{Artikel}\index{Medikament}\index{Lager}
 Alles was eingekauft, gelagert, abgegeben oder rezeptiert werden kann, ist ein \textit{Artikel}.
 Artikel sind in Klassen organisiert, beispielsweise \textit{Medikament},
 oder \textit{MiGeL} oder \textit{Büromaterial}.
 Elexis kann jeden Artikel, den es kennt, als \textit{Lagerartikel} aufnehmen.

 Ein Lagerartikel ist ein Artikel, dessen Bestand monitorisiert wird, und der bei Bedarf auch halbautomatisch nachbestellt werden kann.
 Weitere Informationen zu Artikeln und Lager finden Sie bei der Beschreibung der entsprechenden View (S. \pageref{view:artikel} ff.)

 \section{Mehrere Instanzen gleichzeitig}
 \index{gleichzeitig}
 Sie können Elexis problemlos mehrfach starten und dann in verschiedenen
 Fenstern unterschiedliche Perspektiven oder verschiedene Patienten anzeigen.
 Einzelne Elemente können auch mit Cut\&Paste zwischen den laufenden Instanzen
 ausgetauscht werden.
 Anwendungseispiele:
 \begin{itemize}
   \item Sie arbeiten an einem Patienteneintrag und es kommt ein Telefon
   betreffend eines anderen Patienten. Anstatt Ihre Arbeit zu verlassen, bringen
   Sie die zweite Elexis-Instanz in den Vordergrund und suchen dort den neuen
   Patienten auf.
   \item Die MPA möchte an ihrem Arbeitsplatz Agenda und Patientendaten
   gleichzeitig im Blick haben. Spendieren Sie ihr einen zweiten Monitor (statt
   eines zweiten PC's), schliessen Sie beide Monitore an einer DualHead-fähigen
   Grafikkarte am selben PC an und schieben Sie in jeden Monitor eine eigene
   Instanz von Elexis.
   \item Während Elexis mit einem langwierigen Rechnungsdruck bechäftigt ist,
   möch\-ten Sie nicht untätig herumsitzen. Kein Problem, starten Sie eine zweite
   Instanz von Elexis und arbeiten Sie dort weiter. (Sie könnten natürlich auch
   einen Kaffee trinken oder einen Spaziergang machen).
   \item Sie erstellen einen Brief, möchten aber einzelne Stellen aus einem
   anderen Brief herüberkopieren. Laden Sie in der einen Elexis-Instanz den
   alten Brief, erstellen Sie in der anderen den neuen Brief und kopieren Sie
   das gewünschte mit Cut\&Paste.
 \end{itemize}

\section{Plugins}
Dieses Konzept wird auf Seite \pageref{expl:plugins} genauer besprochen. Hier nur soviel: Elexis ist nach allen Seiten frei erweiterbar. Es gibt nicht nur eine vorgegebene Zahl von \glqq Modulen\grqq{}, sondern tatsächlich können jederzeit auch von Dritten neue Funktionen programmiert werden, von denen zum Zeitpunkt des Programmreleases noch gar nichts bekannt war. Dies geschieht in Form von sogenannten \glqq Plugins\grqq{}. Plugins können beispielsweise für Statisik, Buchhaltung, Import von Labordaten, Anbindung von Apparaten, Export von KG-Daten, neue Abrechnungssysteme, neue Diagnosesysteme usw. programmiert werden.

Ein Elexis-Plugin ist also einfach ein Programm mit im Prinzip beliebigen Fähigkeiten, welches die Eigenschaft hat, mit Elexis zusammenarbeiten zu können.

Es kann weder in diesem Handbuch noch sonstwo eine abschliessende Aufzählung aller Plugins geben, weil niemand wissen kann, welche Plugins von unabhängigen Anwendern bei unabhängigen Programmierern in Auftrag gegeben worden sind. 
\chapter{Menü und Toolbar} 					
	\include{menu}
\chapter{Einstellungen}
	% *******************************************************************************
% * Copyright (c) 2007 by Elexis
% * All rights reserved. This document and the accompanying materials
% * are made available under the terms of the Eclipse Public License v1.0
% * which accompanies this distribution, and is available at
% * http://www.eclipse.org/legal/epl-v10.html
% *
% * Contributors:
% *    G. Weirich - initial implementation
% *
% *  $Id: settings.tex 2651 2007-06-28 19:46:29Z rgw_ch $
% *******************************************************************************
%
% !Mode:: "TeX:UTF-8" (encoding info for WinEdt)

\label{settings}
Die Einstellungen sind alle im selben Dialog zusammengefasst, welcher unter
\textsc{Datei-Einstellungen} erreicht werden kann (Abb. \ref{fig:settingsmain}).
%\usepackage{graphics} is needed for \includegraphics
\begin{figure}[htp]
\begin{center}
  \includegraphics[width=0.6\textwidth]{images/settingsmain}
  \caption{Einstellungs-Dialog}
  \label{fig:settingsmain}
\end{center}
\end{figure}


Wie üblich in Elexis ist der genaue Inhalt dieses Dialogs davon abhängig,
welche Plugins installiert sind. Mit den Reitern auf der linken Seite wählt man
einen Bereich aus, für den man Einstellungen ändern möchte. Wir gehen hier auf
diejenigen Seiten ein, die zur Grundausstattung von Elexis gehören. Grundsätzlich
sollten alle Einstellungen \glqq Ab Werk\grqq{}vernünftige Grundeinstellungen
aufweisen, so dass es zunächst nicht nötig ist, hier etwas zu ändern. Sie
brauchen daher dieses Kapitel auch nicht unbedingt weiterzulesen.

\section{Allgemein}
Auf dieser Seite werden allgemeine Einstellungen für den Programmablauf
definiert. Es sind dies:
\subsection{Einstellungen zum Log}
Das \glqq Log\grqq{} ist das Logbuch eines Programmes. Hier werden verschiedene
Information zum Programmablauf gespeichert, welche z.B. bei der Fehlersuche
nützlich sein können.
\begin{itemize}
  \item Logdatei: Der Ort, an den die Log-Informationen gespeichert werden. Die
  sollte normalerweis eeine Datei \glqq elexis.log\grqq{} in Ihrem
  Datenverzeichnis sein. Der Wert \glqq none\grqq{} ist nur sinnvoll, wenn Sie
  Elexis aus einer Entwicklungsumgebung heraus starten.
  \item Log-Stufe: Wieviele Meldungen ausgegeben werden sollen. Auf Stufe 1
  werden nur die allerschlimmsten Fehler, die einen Programmabbruch erzwingen,
  ausgegeben. Auf Stufe 5 werden sehr viele Meldungen, die nur in speziellenj
  Fällen sinnvoll sind, ins Log geschrieben. Wir empfehlen für den Normalbetrieb
  Stufe 2 oder 3.
  \item Alert-Stufe: Meldungen, die den entsprechenden Schweregrad haben, werden
  nicht nur ins Log geschrieben, sondern gleich am Bildschirm angezeigt.
  Achtung: Wenn Sie hier eine zu hohe Stufe angeben, werden Sie ständig durch
  aufpoppende Meldungsboxen irritiert werden. Wir empfehlen Stufe 1.
  \item Tabellenname für Trace: Trace bedeutet, dass alle Aktionen in einer
  speziellen Tabelle aufgezeichnet werden. Es lässt sich damit später
  nachvollziehen, von welcher Arbeitsstation aus zu welchem Zeitpunkt welche
  Aktion mit Elexis durchgeführt wurde. Dies erlaubt eine sehr genaue Kontrolle
  der Vorgänge, kostet aber natürlich Arbeitsgeschwindigkeit und Speicherplatz.
  Wir empfehlen im Normalfall die Einstellung \glqq none\grqq{}.
  \item Bevorzugte Sprache: Diese Einstellung definiert nicht, welche
  Sprachversion von Elexis ausgeführt wird (Das wird anhand der
  Betriebssystemeinstellungen und ggf. Startparameter entschieden), sondern
  vielmehr, welche Tarmed- und ICD-Versionen etc. importiert werden.
  \item Speicherdauer im Cache: Dies ist eine sehr technische Einstellung. Es
  geht darum, wie lange aus der Datenbank gelesene Objekte gültig bleiben
  sollen, bevor sie erneut gelesen werden. Wenn
  viele Arbeitsstationen im Netz sind, geben Sie hier besser kürzere Zeiten an
  (z.B. 5 Sekunden), wenn Sie von zuhause über eine langsame Internet-Verbindung
  auf Elexis zugreifen, eher eine längere Zeit (z.B. 300 Sekunden).
  \item Aktualisierungsintervall: Nach welcher Zeitspanne soll Elexis jeweils
  seine Views aktualisieren. Wenn beispielsweise die MPA einen Patienten der
  Agenda auf \glqq eingetroffen\grqq{} setzt, dann dauert es maximal soviele
  Sekunden, bis diese Statusänderung auf Ihrem Bildschirm sichtbar ist. Wenn Sie
  zu kurze Zeiten angeben, wird die Netzwerkbelastung unnötig hoch.
\end{itemize}
\section{Anwender}
In diesem Zweig der Einstellungen sind anwenderspezifische Einstellungen
untergebracht. Wenn Sie einheitliche Einstellungen möchten, können Sie auch einen
Einstellungssatz unter einem frei wählbaren Namen speichern und von einem
anderen Anwenderaccount oder Abreitsstation aus wieder unter diesem Namen laden.

Die Buttons \glqq Einstellungen laden von\ldots\grqq{} bzw. \glqq Einstellungen
speichern nach\ldots\grqq{} betreffen hierbei die anwenderspetifischen
Einstellungen(im Wesentlichen alles was im Zweig \textsc{Anwender} der
Einstellungen vorhanden ist), während die Buttons \glqq
Arbeitsplatzeinstellungen \ldots\grqq{} die auf der lokalen Station
gespeicherten Perspektivenlayouts betreffen.
\section{Datenaustausch}
Dies ist eine Sammelkategorie für Einstellungen von Plugins, die Datenaustausch von und nach Elexis anbieten. Ob und welche Einstellungsseiten hier zu finden sind, hängt von den installierten Transport-Plugins ab. 
\section{Datenbank}
Anzeige von Einstellungsdetails der aktuellen Datenbankverbindung
\section{Druckereinstellungen}
Hier kann man für jede Papierart den dazugehörigen Drucker und -Schacht auswählen. Beim Labeldrucker kann man ausserdem einstellen, ob der Druckerauswahldiealog überhaupt jedesmal vor dem Drucken angezeigt werden soll (wenn man z.B. mehrere Labeldrucker hat).
\section{E-Mail}
Diese Einstellungen sind für das Versenden von E-Mails aus Elexis wichtig. Dies wird inbesondere beim automatischen Versenden von Fehlermeldungen verwendet. Dies wird auf Seite \pageref{senderrors} genauer beschrieben.
\section{Gruppen und Rechte}
Dies ist die zentrale Benutzerverwaltung. Auf diesen Einstellungsseiten können Anwender und Mandanten eingerichtet und die Zugriffsrechte verteilt werden. Das Konzept der Gruppen ist auf Seite \pageref{sec:gruppen} genauer erläutert.
Legen Sie zunächst unter \textsc{Gruppen und Rechte} fest, welche Anwendergruppen Sie benötigen. 
Um einen neuen Anwender oder Mandanten einzurichten, müssen Sie diesen zunächst als \glqq Kontakt \grqq{} erfassen, und dort unter Kontakt-Details als Anwender bzw. Mandant kennzeichnen. Dann können Sie unter \textsc{Gruppen und Rechte - Mandanten} dem Mandanten einen Benutzernamen und ein Passwort zuordnen, und angeben, welchen Gruppen er zugehörig sein soll.
Unter \textsc{Gruppen und Rechte - Anwender} können Sie dasselbe für Anwender angeben, ausserdem noch, für welchen Mandant dieser Anwender normalerweise tätig ist. 
Unter \textsc{Gruppen und Rechte - Zugriffsteuerung} können für jede Gruppe und jeden Anwender einzeln Rechte zugeordnet werden.  (S. \ref{sec:gruppen}).
\section{Laborwerte}
Hier können die in der Praxis benötigten Laborparameter definiert werden. Dies kann manuell geschehen, oder, bei Laborimport-Plugins können Laboritems auch automatisch mit den vom Labor gelieferten Angaben erstellt werden.
Jedes Laboritems ist durch folgende Eckdaten gekennzeichnet:
\begin{itemize}
\item{Einen Namen}
\item{Ein Kürzel}
\item{Das Labor, von dem es stammt}
\item{Den Normbereich, gegeben durch die Methode, z.T. auch geschlechts- alters- zyklusabhängig}
\item{Eine Gruppe, unter der des aufgelistet wird (z.B. Hämatologie)}
\item{Eine Sequenznummer, die angibt, an welcher Stelle innerhalb der Gruppe es einsortiert wird.}
\item{einen Typ (numerisch, absolut, text)}
\end{itemize}

Jedes Laborresultat ist durch ein solches Item, ein Datum und einen Patienten eindeutig identifiziert. 
Es kann deswegen durchaus mehrere Items für ein- und denselben Parameter geben. Beispielsweise kann es ein Item \textsc{Vitamin B12} von verschiedenen Labors geben, welche nicht zwingend denselben Normbereich haben müssen.
 
Mit \textsc{Neuer Laborparameter} können Sie ein neues Item erstellen und die oben genannten Angaben eingeben. Hier ist es sehr wichtig, dass Sie sich im Voraus genau überlegen, welche Laborparameter Sie benötigen, und wie Sie diese gruppiert haben wollen.

Sie können die Liste der Tabelle durch Klick auf die Spaltenköpfe umsortieren.
\section{Leistungscodes}
Dies ist wieder eine Sammelrubrik, die je nach vorhandenen Plugins für die Leistungsabrechnung unterschiedlich gefüllt sein kann. In der Schweiz ist hier standardmössig Labortarif und Tarmed vorhanden. Diese sind auf Seite \pageref{arzttarife} genauer erklärt.
\section{Textverarbeitung}
Auch die verwendete Textverarbeitung für Briefe etc. ist in Elexis ja durch Plugins frei definierbar. Welche Textverarbeitung verwendet werden soll, kann hier eingestellt werden. Diese Einstellung sollte normalerweise nicht mehr verändert werden, wenn erste Dokumente erstellt wurden, da diese sonst eventuell nicht mehr ohne weiteres lesbar wären. Wir empfehlen, unter Windows das Plugin NOAText und unter Linux Office-Wrapper zu verwenden. 
\section{Update-Einstellungen}
Elexis kann automatisch im Internet nach neuen Versionen des Kernprogramms und aller aktuell installierten Plugins suchen und diese ggf. auch automatisch herunterladen und installieren. Dieses Verhalten kann hier eingestellt werden.
\begin{itemize}
\item{update-site} Hier müssen Sie angeben auf welcher Site Elexis nach updates sehen soll. Verwenden Sie normalerweise die Voreinstellung.
\item{Alle (0-nie) Tage automatisch suchen} Wenn Sie hier einen Wert ungleich null eintragen, dann gibt dieser Wert die Zahl der Tage an, die Elexis vor einem automatischen Update verstreichen lässt. Wenn der Wert null ist, dann wird Elexis nie von allein einen update-Vorgang starten (Dieser kann aber trotzdem immer manuell im Menu \textsc{Datei-Update} ausgelöst werden).
\item{Verzeichnis für Zwischenspeicherung} Dies gibt das Verzeichnis an, in dem Elexis hehruntergeladene Updates zwischenspeichern kann bis zum Programmende.
\end{itemize} 
Da Elexis nicht Teile von sich selber im laufenden Betrieb austauschen kann, geht das Update so vor: Zu ersetzende Programmteile werden zunächst in einem Verzeichnis zwischengespeichert. Erst beim nächsten Programmende werden diese Änderungen eingespielt und stehen beim nächsten Programmstart zur Verfügung. Sie werden deshalb eventuell beim Programm beenden nach einem Update eine Meldung sehen, die besagt, dass Sie den PC noch nicht ausschalten sollen, bis Elexis den Update-Vorgang fertiggestellt hat.

\chapter{Views des Kernsystems}
	\include{views/einfuehrung}
	% *******************************************************************************
% * Copyright (c) 2007 by Elexis
% * All rights reserved. This document and the accompanying materials
% * are made available under the terms of the Eclipse Public License v1.0
% * which accompanies this distribution, and is available at
% * http://www.eclipse.org/legal/epl-v10.html
% *
% * Contributors:
% *    G. Weirich - initial implementation
% *
% *  $Id: stammdaten.tex 3569 2008-01-21 17:55:22Z rgw_ch $
% *******************************************************************************
% !Mode:: "TeX:UTF-8" (encoding info for WinEdt)

\section{Stammdaten-Views}

\subsection{Patienten}
\index{Patientenliste}
Die Patientenliste dient sowohl der Anzeige existierender Patienteneinträge, als
auch dem Erfassen neuer Einträge. Die Liste zeigt all diejenigen Kontakte an,
die als Patient markiert sind.
\begin{figure}[ht]
	\includegraphics{images/patlistview}
	\caption{Patientenliste}
	\label{fig:patlist}
\end{figure}
Die Eingabefelder oben (Name, Vorname, Geburtsdatum) dienen dem Filtern der
Liste gemäss den gewünschten Parametern.
\begin{itemize}
  \item Bei Name und Vorname gilt:
	\begin{itemize}
      \item Wenn Sie mindestens zwei Buchstaben eingeben, erscheinen in der
      Liste nur noch diejenigen Einträge, die mit diesen Buchstaben \textit{beginnen}.
      \item Wenn Sie das Zeichen \% und mindestens zwei weitere Buchstaben
      eingeben, dann erscheinen in der Liste diejenigen Einträge, die diese
      Zeichenfolge \textit{enthalten}.
    \end{itemize}
  \item Beim Geburtsdatum gilt:
	\begin{itemize}
      \item Wenn Sie mindestens 3 aufeinanderfolgende Ziffern eingeben, dann wird
      die Zahl als Jahreszahl interpretiert und es werden diejenigen Patienten
      ausgewählt, die das entsprechende \textit{Geburtsjahr} haben.
      \item Wenn Sie zwei Ziffern, gefolgt von einem Punkt und ggf. weitere 2
      Ziffern eingeben, dann werden diejenigen Patienten angezeigt, die den
      entsprechenden \textit{Geburtstag} und ggf. \textit{Geburtsmonat} haben.
      Beachten Sie bitte, dass Sie Tag und Monat zweistellig eingeben müssen,
      also z.B. 04.05. und nicht etwa 4.5.
     \end{itemize}
\end{itemize}
Wenn keine Einträge existieren, die den eingegebenen Filterbedingungen
entsprechen, dann wird in der Liste angezeigt: \glqq keine Daten\grqq.
\subsubsection{Toolbar}
\begin{itemize}
  \item Mit der Taste \glqq Neu\grqq{} (s. Abb. \ref{fig:patlist}) können Sie
  einen neuen Patienten erfassen. Klick
  auf diesen Knopf öffnet die Patienteneingabe-Dialogbox. Diejenigen Felder, die
  Sie bereits eingegeben haben, sind vorgegeben, die anderen können Sie soweit
  eingeben, wie sie im Moment bekannt sind. Mit Klick auf \glqq OK\grqq{}wird
  der neue Patienteneintrag angelegt. Bei Klick auf \glqq Abbrechen\grqq{}werden
  die eingegebenen Daten verworfen und es wird kein neuer Eintrag erstellt.
  Falls ein neuer Eintrag erstellt werden soll, und bereits ein Eintrag mit
  gleichen Daten existiert, dann erfolgt eine Rückfrage.

  \item Mit der Taste \glqq Filter\grqq{} (s. Abb. \ref{fig:patlist}) können Sie
  die Liste nach
  verschiedenen Kriterien filtern (s. Abb. \ref{fig:patlistfilter}).
	\begin{figure}[ht]
    	\includegraphics{images/patlistfilter}
    	\caption{Filterbedingungen eingeben}
    	\label{fig:patlistfilter}
    \end{figure}
	\textbf{Achtung}: Der Filter rastet ein und wird erst wieder aufgehoben, wenn
	Sie erneut auf den Filter-Knopf klicken. Solange er eingerastet ist, wirken
	alle anderen Eingaben nur auf die bereits gefilterte Auswahl.
\end{itemize}
\subsubsection{View-Menü}
Das lokale Menü (s. Abb. \ref{fig:patlist}) enthält folgende Einträge:
\begin{itemize}
  \item Patient löschen (sofern der angemeldete Anwender die hierfür notwendigen
  Rechte hat): Damit kann ein Patienteneintrag definitiv und unwiderruflich aus
  der Datenbank gelöscht werden. Dies ist nur dann möglich, wenn zu diesem
  Patienten keine Fälle (mehr) existieren.
  \item Liste filtern: Dies öffnet ebenfalls die Filter-Dialogbox (s.o.)
\end{itemize}

\subsubsection{Kontextmenü}
Das Kontextmenü erscheint, wenn Sie auf einem Pa\-tien\-ten\-eintrag mit der rech\-ten
Maus\-taste klicken. Es enthält fol\-gende Ein\-träge:
\begin{itemize}
  \item Patient löschen (s. oben)\footnote{Sie benötigen dazu das Recht \textsc{Löschen/Kontakt} (Vgl. \ref{sec:gruppen})}
  \item KG exportieren. Falls ein Export-Plugin installiert ist, wird die KG des
  aktuell markierten Patienten über dieses Plugin exportiert. Falls mehrere
  Export-Plugins definiert sind, erscheint zunächst eine Dialogbox, mit der sie
  das ge\-wünschte Ziel bzw. Format auswählen können.\footnote{Sie benötigen dazu das Recht \textsc{Daten/Kontakt/exportieren}}
\end{itemize}


\subsection{Patient-Detail}
\index{Patient!Detailangaben}
Diese View (Abb. \ref{fig:patdetail} zeigt Details des momentan ausgewählten
Patienten resp. der momentan ausgewählten Patientin an
 %\usepackage{graphics} is needed for \includegraphics
\begin{figure}[htp]
\begin{center}
  \includegraphics[width=0.9\textwidth]{images/patdetail}
  \caption{Patient Detailansicht}
  \label{fig:patdetail}
\end{center}
\end{figure}
Alle Felder können durch einfaches Überschreiben geändert werden, sofern der
aktuell eingeloggte Anwender die dazu erforderlichen Rechte besitzt. Eine
Änderung wird in dem Moment gespeichert, in dem ein Feld wieder verlassen wird.
(Explizites Speichern ist in Elexis nie notwendig).

Die Felder im oberen Block sind alle einzeilige Textfelder und können direkt geändert
werden, bis auf das Feld \glqq Konto\grqq{}, welches nicht direkt beschreibbar
ist. Dieses Feld stellt den Saldo aller Forderungen an diesen bzw. Zahlungen von
diesem Patienten dar. Wenn der aktuell eingeloggte Anwender Verrechnungs-Rechte
besitzt, kann er den blauen Text \glqq Konto\grqq{} anklicken, dann öffnet sich
ein Dialog, in dem einzelne Buchungen eingegeben werden können.

\textbf{Achtung}: Normalerweise erfolgen Buchungen automatisch durch Erstellen
von Rechnungen und Einlesen von ESR-Files. Manuelle Buchungen können zu
Inkonsistenzen in der Buchhaltung führen. Führen Sie also nur dann manuelle
Buchungen durch, wenn Sie sich über die Konsequenzen exakt bewusst sind.

Das Feld \glqq Anschrift\grqq{} zeigt die Postanschrift des Patienten an. Diese
kann durch Klick auf den blauen Text \glqq Anschrift \grqq{}geändert werden.

Die darunterstehenden Felder sind alle aufklappbar: Standardmässig ist nur der
Titel sichtbar, durch Klick darauf öffnet sich das Feld.
\begin{itemize}
  \item Das Feld \glqq Zusatzadressen\grqq{}dient dazu, Kontakte, die in
  irgendeiner Beziehung zum Patienten stehen, zu erfassen. Beispielsweise
  Angehörige, Ämter, weitere Ärzte etc. Klick auf \glqq Hinzu\grqq{} öffnet eine
  Kontaktauswahl-Box, aus der die gewünschte Person oder Organisation ausgewählt
  werden kann. Danach erscheint eine Eingabebox, in der die Beziehung des eben
  ausgewählten Kontakts zum Patienten beschrieben werden kann. \\
  Mit Rechtsklick auf einen Eintrag in diesem Feld öffnet sich ein Kontextmenü,
  mit dem man den vollständigen Eintrag anzeigen, oder den Eintrag entfernen kann.
  \item Die Felder \glqq Diagnose\grqq, \glqq Persönliche Anamnese\grqq{},
  \glqq Allergien\grqq{}, \glqq Risiken\grqq{} und \glqq Bemerkungen \footnote{Falls in \textit{Bemerkungen} irgendwo :VIP: (mit den Doppelpunkten) steht, so wird der betreffende Pat. in roter Schrift angezeigt.}\grqq{}
  können direkt beschrieben werden und werden wie gewohnt sofort beim Verlassen
  gespeichert.
  \item Das Feld \glqq Fixmedikation\grqq{}entspricht der View Fixmedikation.
  \item Das Feld \glqq Reminders\grqq{} zeigt die Reminders zum aktuellen Patienten.
\end{itemize}
\subsection{Kontakte}
Diese View (Abb. \ref{fig:kontaktlist}) zeigt eine Liste aller in Elexis
vorhandenen Kontakte an. Ein Kontakt
ist jede Person oder jede Organisation, welche in irgendeiner Beziehung zu
unserer Praxis steht. Das sind beispielsweise Patienten, Kollegen, Spitäler,
Versicherungen, Labors, Lieferanten usw.
%\usepackage{graphics} is needed for \includegraphics
\begin{figure}[htp]
\begin{center}
  \includegraphics[width=0.9\textwidth]{images/kontaktlistview}
  \caption{Kontaktliste-View}
  \label{fig:kontaktlist}
\end{center}
\end{figure}
Mit Klick auf das Briefumschlag-Symbol rechts oben können Sie eine
Adressetikette für den betreffenden Kontakt ausdrucken.

\subsection{Kontakt-Detail}
Hier werden die Details zum aktuell ausgewählten Kontakt angezeigt und können
geändert werden (Abb. \ref{fig:kontaktdetail}).
%\usepackage{graphics} is needed for \includegraphics
\begin{figure}[htp]
\begin{center}
  \includegraphics[width=0.9\textwidth]{images/kontaktdetail}
  \caption{Kontakt Detailview}
  \label{fig:kontaktdetail}
\end{center}
\end{figure}
In den Checkboxen der obersten Zeile können Sie den Typ des betreffenden
Kontakts festlegen. Beachten Sie, dass ein Kontakt auch mehrere Typen haben kann
(Beispielsweise kann jemand Anwender und auch Patient sein). Hingegen kann
ein Kontakt natürlich nur entweder eine Organisation oder aber eine Person sein.
Achten Sie darauf, dies und bei Personen auch das Geschlecht (m oder w) korrekt
zu erfassen, da Textformatvorlagen diese Informationen auswerten, um die
korrekten Formulierungen auszuwählen.

In der untersten Zeile steht die Postanschrift des betreffenden Kontakts. Dies
ist die Adresse, wie sie beispielsweise im Adressfeld von Briefen oder
Rechnungen oder auf Adressetiketten erscheinen soll. Mit Klick auf das blaue
Wort \glqq Anschrift\grqq{}öffnet sich die Anschrifteingabe-Dialogbox (Abb.
\ref{fig:anschrift}), wo Sie beliebigen Text eingeben können. (Klick auf den
Button \glqq Postanschrift\grqq{} erstellt eine Standard-Anschrift aus den
vorhandenen Adressangaben)


\begin{figure}[htp]
\begin{center}
  \includegraphics{images/anschrifteingabe}
  \caption{Anschrift-Eingabe}
  \label{fig:anschrift}
\end{center}
\end{figure}
\bigskip
\pagebreak[3]
\subsection{Artikel}
\label{view:artikel}
\index{Artikel}
Ein \glqq Artikel\grqq{}ist jedes Objekt, das auf Lager genommen und/oder
abgegeben werden kann. Es gibt einerseits vordefinierte Artikel (z.B. die Liste
aller zugelassenen Medikamente), andererseits auch Eigenartikel. Elexis kann den
Lagerbestand von Lagerartikeln verwalten und halbautomatisch Bestellungen zur Neige
gehender Artikel vornehmen.


\subsection{Artikelliste}
In Abb. \ref{fig:artikel} ist eine Artikelauswahl-Liste und die
Artikeldetaildarstellung nebeneinander zu sehen.

 %\usepackage{graphics} is needed for \includegraphics
\begin{figure}[htp]
\begin{center}
  \includegraphics[width=0.9\textwidth]{images/artikelview}
  \caption{Artikel-View}
  \label{fig:artikel}
\end{center}
\end{figure}
Die Liste links können Sie in gewohnter Weise filtern, indem Sie einige
Buchstaben des gewünschten Artikelnamens eingeben.
In der Detailansicht sehen Sie Einzelheiten zum gerade ausgewählten Artikel. (In
Perspektiven, wo die Liste allein dargestellt ist, können Sie mit der rechten
Maustaste und \glqq Bearbeiten\grqq{} zur Detailansicht gelangen.)

\index{Lager}
Ein Artikel wird dadurch zum Lagerartikel, dass Sie ihm einen Mindestbestand
grösser als Null zuweisen. Geben Sie ausserdem einen Höchstbestand höher als der
Mindestbestand ein und weisen Sie dem Feld \glqq Istbestand\grqq{} den korrekten
Wert zu. Elexis wird bei einer halbautomatischen Bestellung von jedem Artikel,
dessen Istbestand unter dem Mindestbestand ist, soviele Exemplare bestellen, um
auf den Höchstbestand zu kommen.

Bei manchen Artikeltypen wird üblicherweise nicht eine ganze Verpackungseinheit
auf einmal abgegeben, beispielsweise Ampullen. Hierfür sind die Felder \glqq
Stück pro Packung\grqq{}und \glqq Stück pro Abgabe\grqq{}vorgesehen. Angenommen
ein Artikel wird in Packungen zu 10 Stück eingekauft, aber einzeln abgegeben.
In diesem Fall können Sie bei Stück pro Abgabe eine 1 setzen, bei Stück pro
Packung eine 10. Wenn dieser Artikel dann einem Patienten verrechnet wird, dann
wird automatisch 1/10 des Verpackungs-Verkaufspreises berechnet und auch nur
1/10 einer Packung aus dem Lager ausgebucht.

Die Angabe \glqq Aktuell an Lager\grqq{} meint dann die Zahl der einzelnen
Artikel, während \glqq Aktuell Pck. an Lager\grqq{} für die Zahl der
unangebrochenen Packungen steht.


\subsection{Lager und Bestellung}
\index{Artikel!bestellen}
Wie oben beschrieben, kann Elexis Ihr Warenlager halbautomatisch bewirtschaften.
Wann immer Sie einem Patienten einen Artikel verrechnen, wird dieser Artikel
automatisch aus dem Lagerbestand ausgebucht. Sobald der Bestand eines
Lagerartikels unter den von Ihnen definierten Mindestbestand fällt, \glqq
weiss\grqq{} Elexis, dass dieser Artikel nachbestellt werden muss. Nebst dieser
automatischen Erkennung können Sie selbstverständlich Bestellungen auch manuell
erstellen und/oder ändern.

Diese Funktionen sind in der View \glqq Bestellung\grqq{} erreichbar (s. fig.
\ref{fig:bestellungen}).
 %\usepackage{graphics} is needed for \includegraphics
\begin{figure}[htp]
\begin{center}
  \includegraphics[width=0.9\textwidth]{images/bestell1}
  \caption{Bestellungen - View}
  \label{fig:bestellungen}
\end{center}
\end{figure}


Links finden Sie die schon bekannten Artikelauswahlfenster für alle
Artikelkategorien, für die Sie Plugins haben (Normalerweise Medikamente,
Medicals, MiGeL und Eigenartikel). Rechts ist das Feld Bestellung, welches
anfangs leer ist. Sie haben nun folgende Möglichkeiten:
\begin{itemize}
  \item Mit Klick auf das Zauberstab-Symbol werden automatisch diejenigen
  Artikel der Bestellung zugefügt, von welchen weniger als der Mindestbestand an
  Lager ist. Es werden jeweils soviele bestellt, dass der für diesen Artikel
  definierte Höchstbestand erreicht wird.
  \item Sie können aus einem der Fenster links Artikel in die Bestellung
  herüberziehen.
	\item Sie können auf einen der Artikel in der Bestelliste mit der rechten
	Maustase klicken, und den Artikel aus der Liste entfernen oder die Zahl ändern.
	
	\item Sie können die Bestellung erst mal abspeichern und später weiterbearbeiten.
	
	\item Sie können eine früüher gespeicherte Bestellung wieder laden.
	\item Sie können die Bestellung ausdrucken. Dafür ist eine System-Textvorlage (s. S.
	\pageref{textvorlagen}) namens \glqq Bestellung\grqq{} notwendig, welche an
	einer Stelle den Platzhalter [Bestellung] enthält (s. Abb. \ref{fig:bestell2}).
	\item Last but not least können Sie, falls Sie ein entsprechendes Plugin für
	Ihren Lieferanten haben, die Bestellung direkt via Internet oder Modem
	absenden. Ein entsprechendes Plugin für Galexis ist bereits verfügbar, weitere
	werden entwickelt.
\end{itemize}
\begin{figure}[hb]
  % Requires \usepackage{graphicx}
  \includegraphics{images/bestell2}\\
  \caption{Ausschnitt aus der Vorlage Bestellung}\label{fig:bestell2}
\end{figure}

\subsection{codes}
Codes




	% *******************************************************************************
% * Copyright (c) 2007 by Elexis
% * All rights reserved. This document and the accompanying materials
% * are made available under the terms of the Eclipse Public License v1.0
% * which accompanies this distribution, and is available at
% * http://www.eclipse.org/legal/epl-v10.html
% *
% *  $Id: abrechnung.tex 2932 2007-07-29 06:21:06Z rgw_ch $
%
%*******************************************************************************
% !Mode:: "TeX:UTF-8" (encoding info for WinEdt)

\section{Abrechnungsbezogene Views}
\subsection{Konsultationen nach Datum}
\begin{wrapfigure}{l}{7cm}
\includegraphics{images/konnd}
\caption{Konsultation nach Datum}
\label{fig:konnd}
\end{wrapfigure}
Diese View (s. Abb. \ref{fig:konnd}) dient dazu, Ihnen alle in einem bestimmten
Zeitraum stattgefundenen
Konsultationen aufzulisten. Sie können in den Datumsfeldern oben das Start- und
das Enddatum des gewünschten Zeitraums angeben.

Der Knopf \glqq nur offene\grqq kann betätigt werden, um in die Auflistung nur
solche Konsultationen aufzunehmen, für die noch keine Rechnung erstellt wurde.

Wenn Sie die Daten oder den Auswahlknopf geändert haben, erscheint die neu
berechnete Liste erst nach Klick auf den Button \glqq Liste neu berechnen\grqq.

Im unteren Abschnitt der View sehen Sie die Gesamtzahl der Konsultationen im
gewählten Zeitraum, sowie die (vom Abrechnungssystem her vorgegebene)
verrechnete Zeit und den verrechneten Betrag. Im Feld darunter sehen Sie
dieselben Angaben für die aktuell markierte Konsultation.

Sie können diese View also auch verwenden, um Abends kurz die Konsultationen des Tages
durchzugehen um unverrechnete oder falsch verrechnete zu korrigieren. Die Liste
kann auch zur einfacheren Lesbarkeit ausgedruckt werden (\textsc{ViewMenu-Liste drucken})

\clearpage

\subsection{Konsultationen zum Verrechnen}
\index{Abrechnung} Diese View (s. fig. \ref{fig:konsv}) dient dazu, diejenigen
Konsultationen
auszuwählen, von welchen eine Rechnung erstellt werden soll. Es werden dabei nur
die Konsultationen des aktuellen Mandanten angezeigt.
\begin{figure}[hb]
\includegraphics{images/konsv}
\caption{Konsultation zur Verrechnung auswählen}
\label {fig:konsv}
\end{figure}
Hierzu gibt es folgende Möglichkeiten:
\begin{itemize}
  \item Automatische Auswahl (Zauberstab-Icon): Dabei werden die Konsultationen nach bestimmten Regeln automatisch ausgewählt und in die Auswahlliste
  übertragen.
  \item Patientennamen aus der Liste in die Auswahl ziehen: Dadurch werden alle
  Konsultationen aller Fälle des gewählten Patienten zur Abrechnung markiert.
  \item Fälle aus der Liste in die Auswahl ziehen: Dadurch werden alle
  Konsultationen der gewählten Fälle zur Abrechnung markiert.
  \item Konsultationen aus der Liste in die Auswahl ziehen: Dadurch werden nur
  die gewählten Konsultationen zur Abrechnung vorgemerkt.
\end{itemize}
Bei allen Methoden können Sie die Auswahl nachträglich noch beliebig ändern. Sie
können weitere Elemente zufügen, oder Sie können (nach Rechtsklick auf ein
Element in der Auswahl) Elemente entfernen, oder Sie können die ganze Auswahl
wieder löschen. Zu diesem Zeitpunkt sind noch keinerlei Änderungen der Daten
erfolgt.

Wenn Sie die Auswahl fertig erstellt haben, können Sie auf \glqq Rechnungen
erstellen\grqq klicken, dann werden Rechnungen für alle in der Auswahl
befindlichen Elemente erstellt. Dabei werden immer alle Konsultationen, die zu
einem Fall gehören, zusammengefasst. Wenn von einem Patienten also mehrere Fälle
in der Auswahl sind, werden auch mehrere Rechnungen erstellt.

\clearpage 

\subsection{Rechnungen}
\begin{figure}[ht]
  % Requires \usepackage{graphicx}
  \includegraphics[width=0.8\textwidth]{images/rechnungsview}\\
  \caption{Rechnungen-View}\label{fig:rechnungen}
\end{figure}

In dieser View (Abb. \ref{fig:rechnungen}) sehen Sie die erstellten Rechnungen. Eine Rechnung hat immer einen bestimmten Status:
\begin{description}
    \item [Offen] Unmittelbar nach dem Erstellen.
    \item{Offen und gedruckt} Die Rechnung wurde mindestens einmal ausgegeben (über den Drucker oder eine andere Exportmethode). Ab diesem Zeitpunkt beginnt die Zahlungsfrist zu laufen. (Elexis kann allerdings nicht feststellen, ob beispielsweise der Drucker die Rechnung nicht korrekt ausgedruckt hat, oder ob sie nicht abgeschickt wurde. Deshalb liegt in diesem Punkt eine potentielle Fehlerquelle)
    \item[Zahlungserinnerung] Die Zahlungserinnerung wurde erstellt, aber noch nicht ausgedruckt
    \item[ZE gedruckt] Die Zahlungserinnerung wurdr ausgedruckt
    \item [2. Mahnung erstellt, 2. Mahnung gedruckt, 3. Mahnung erstellt, 3. Mahnung gedruckt]: analog
    \item[Teilweise bezahlt] Es ist (mindestens) eine Zahlung eingebucht, welche aber nicht den ganzen Rechnungsbetrag abdeckt.
    \item[bezahlt] Der Rechnungsbetrag wurde (in einer oder mehreren Buchungen) vollständig bezahlt
    \item [zuviel bezahlt] Auch das kommt vor.
    \item [Teilverlust] Ein Teil de Rechnungsbetrages wird abgeschrieben (Im Gegensatz zu \glqq Teilweise bezahlt\grqq{} rechnen Sie hier nicht mehr mit einer weiteren Zahlung)
    \item [Totalverlust] Der Rechnungsbetrag wird komplett abgeschrieben
    \item [In Betreibung] Genau das.
    \item [Storniert] Eine einmal erstellte Rechnung kann nicht mehr gelöscht werden. Das muss so sein, weil sonst die Situation möglich wäre, dass jemand eine nicht mehr existierende Rechnung reklamiert oder das Rückfragen zu einer inexistenten Rechnung kämen. Wenn eine Rechnung aus irgendeinem Grund ungültig ist (Fehler, Erlassen des Betrags etc.), dann muss sie stattdessen storniert werden. Stornieren hat in allen praktischen Belangen denselben Effekt wie löschen, ausser, dass die Rechnungsnummer vergeben bleibt und dass die Rechnung später wieder betrachtet werden kann.
    \item [fehlerhaft] Wenn ein Rechnungsausgabemodul feststellt, dass eine Rechnung fehlerhaft ist (beispielsweise könnte das TrustX-Modul monieren, dass nicht alle EAN-Nummern angegeben sind), dann erhält die betreffende Rechnung den Status fehlerhaft und kann so korrigiert werden.
    \item [zu drucken] Diese Einstellung findet alle Rechnungen, die offen, aber noch nicht gedruckt sind (also auch undegdurckte Mahnungen etc.)
\end{description}
Um die Rechnungen mit einem bestimmten Status anzuzeigen, wählen Sie diesen Status in der Combox links oben aus(S. Abb. \ref{fig:rechnungen}).
Um nur die Rechnungen eines betsimmten Patienten anzuzeigen, klicken Sie auf die bleuae Schrift \glqq Patient\grqq{}. Es öffnet sich die bekannte Kontaktauswahl-Dialogbox. Wählen Sie dort einen Patienten aus und klicken Sie ok, oder klicken Sie auf Abbrechen, um wieder alle Patienten anzuzeigen. Die Felder \glqq Datum von\grqq{} und \glqq Datum bis\grqq{} dienen dazu, nur Rechnungen auszuwählen, welche zwischen diesen Daten erstellt wurden. Wenn Sie auf die blaue Schrift Datum von oder Datum bis klicken, wechselt diese auf \glqq Status von\grqq{} bzw. \glqq Status bis\grqq{}. In diesem Fall werden diejenigen Rechnungen ausgewählt, bei denen die letzte \textit{Statusänderung} zwischen den angegebenen Daten liegt.
Das Feld \glqq Rn. Nummer\grqq{} ganz rechts dient schliessklich dazu, nur eine ganz bestimmte Rechnung mit dieser NUmmer herauszusuchen.

Wenn Sie die Felder wie gewünscht eingestellt haben, klicken Sie auf \glqq neu einlesen \grqq{}, dann wird die Liste anhand Ihrer Kriterien neu aufgebaut.
Unten sehen Sie jeweils die Zahl der mit diesen Kriterien vorhandenen Rechnungen, sowie die Summen.
\subsubsection{Rechnungen ändern}
Wenn Sie eine Rechnung der Liste mit der rechten Maustaste anklicken, können Sie diese Rechnung ändern:
\begin{itemize}
\item Ausgeben: Die Rechnung einzeln ausgeben (s. unten)
\item Buchung/Zahlung hinzufügen: Hier können Sie manuell Buchungen eingeben, z.B. wenn eine Barzahlung oder Anzahlung erfolgt ist. (Normalerweise erfolgen Buchungen via ESR automatisch).
\item Gebühr zuschlagen: Manuell z.B. Mahngebühr zufügen
\item Status ändern: Hier klann der Rechnungsstatus manuell geändert werden. Die meisten Statusänderungen erkennt Elexis automatisch. So werden z.B. beim Einlesen einer ESR-Datei von der Bank alle bezahlten Rechnungen automatisch auf \glqq bezahlt\grqq{} gesetzt etc. Manche Statusänderungen können aber nur manuell gemacht werden. Zum Beispiel kann Elexis den Unterschied zwischen \glqq Teilweise bezahlt\grqq{} und \glqq Teilverlust\grqq{} nicht automatisch erkennen, weil dies ja eine bewusste Entscheidung des Gläubigers ist. Dasselbe gilt für \glqq In Betreibung\grqq{} und \glqq Totalverlust\grqq{}.
    Von diesen Fällen abgesehen, sollten Sie aber vorsichtig sein mit manuellen Statusänderungen, da hierbei beispielsweise keine Buchungskorrekturen erfolgen.
\item Mahnstufe erhöhen: Hierdurch wird die Mahnstufe jeweils um eins erhöht bis max. Dritte Mahnung.
\item Stornieren: Damit wird die markierte Rechnung storniert. Man hat dabei die Möglichkeit, Behandlungen wieder freizugeben (z.B. wenn die Rechnung fehlerhaft war und neu erstellt werden soll), oder blockiert zu lassen (Wenn diese Behandlungen definitiv nicht verrechnet werden sollen).
\end{itemize}

\subsubsection{Rechnungen ausgeben}
Mit dem Button \glqq Rechnungen ausgeben\grqq{} werden alle markierten Rechnungen ausgegeben. (Um eine Rechnung zu markieren, klicken Sie mit der linken Maustaste auf diese. Um mehrere Rechnungen zu markieren, klicken Sie mit gedrückter Ctrl (bzw. Mac-) Taste auf die gewünschten Rechnungen. Um eine ganze Reihe zu markieren, klicken Sie zuerst auf die erste, dann mit gedrückter Shift-Taste auf die letzte Rechnung aus der Reihe.) Es wird also \textit{nicht} die ganze Liste ausgegeben, sondern nur die markierten Rechnungen!


Die möglichen Ziele der Rechnungsausgabe hängt von den installierten Abrechnungs-Plugins ab. Es kann zum Beispiel ein Drucker sein, der Tarmed-Rechnungen ausdruckt. Es kann aber auch eine XML-Datei oder direkt ein Trust-Center sein. Nähere Angaben dazu finden Sie in den entsprechenden Kapiteln (Tarmed: S. \pageref{arzttarife}, Trustx: S. \pageref{trustx})
\bigskip
Mit Klick auf das Zauberstab-Icon schliesslich setzen Sie die Mahnungen-Automatik in Gang. Diese wählt Rechnungen anhand der im unten rechts angezeigten Feld aus, erhöht die Mahnstufe, fügt wie gewünscht Gebühren zu und fasst diese Rechnungen als \glqq zu drucken\grqq{} zusammen.

\subsection{Konto}
\index{Konto}In dieser View sehen Sie alle Kontobewegungen eines bestimmten Patienten.
Rechnungen werden als negative, Zahlungen und Storno als positive
Buchungen erfasst, so dass Sie einfach über mehrere Rechnungen und Zahlungen
hinweg erkennen können, wo Sie finanziell mit dem betreffenden Klienten stehen.

\subsection{Konto-Liste}
Diese Liste zeigt alle Kontobewegungen insgesamt an.

\subsection{Leistungen}





	% *******************************************************************************
% * Copyright (c) 2007 by Elexis
% * All rights reserved. This document and the accompanying materials
% * are made available under the terms of the Eclipse Public License v1.0
% * which accompanies this distribution, and is available at
% * http://www.eclipse.org/legal/epl-v10.html
% *
% * Contributors:
% *    G. Weirich - initial implementation
% *
% *  $Id: konsviews.tex 2925 2007-07-28 05:21:25Z rgw_ch $
% *******************************************************************************
% !Mode:: "TeX:UTF-8" (encoding info for WinEdt)


\section{Konsultationsbezogene Views}

\subsection{Fälle}
Diese View (Abb. \ref{fig:faelle2} listet alle für den aktuell selektierten
Patienten existierenden Fälle. \index{Fall-Liste}
\begin{wrapfigure}{l}{8cm}
  \includegraphics{images/faelleview}
  \caption{Fälle - View}
  \label{fig:faelle2}
\end{wrapfigure}

Das Symbol links von der Fallbezeichnung gibt an, ob alle für die Verrechnung
des Falles notwendigen Daten vorhanden sind: Wenn es grün ist, sollte die
Rechnungserstellung möglich sein, wenn es rot ist, fehlen noch eine oder mehrere
Angaben. \textit{Welche} Angaben mindestens notwendig sind, hängt von der Art
des Falles ab. So ist für Fälle, die nach dem KVG abgerechnet werden, die Angabe
eines Rechnungsempfängers, eines Versicherers und der Versichertennummer
notwendig. Bei Fällen, die nach UVG abgerechnet werden, muss eine Fallnummer
vorhanden sein.
\index{konsultationen!filtern}
Klick auf das Filtersymbol in der Titelzeile der View führt dazu, dass nu noch diejenigen Konsultationen in der Konsultationsliste (siehe \ref{view:konsultationen}) angezeigt werden, welche zum gerade ausgewählten Fall gehören. Wenn ein anderer Fall angeklickt wird, wird die Liste erneut gefiltert. Erneuter Klick auf das Filter-Symbol schaltet den Filter wieder aus.

Rechtsklick auf einen Fall öffnet dessen Kontextmenü. Dieses enthält die
folgenden Punkte:
\begin{itemize}
  \item {Fall löschen}. Dies ist nur möglich, wenn Sie die dazu notwendigen Rechte
  haben, und wenn zu diesem Fall keine Konsultationen mehr existieren.
  \item {Fall bearbeiten}. Dies öffnet eine weitere View, in der Details zum
  aktuell ausgewählten Fall eingegeben werden können.
  \item {Fall wieder öffnen}. Damit kann man einen bereits
  geschlossenen\footnote{Ein Fall ist dann geschlossen, wenn ein End-Datum
  eingegeben wurde. Zu einem geschlossenen Fall können keine Konsultationen mehr
  hinzugefügt werden.} Fall wieder öffnen.
  \item {Rechnung erstellen}. Hiermit lässt sich über alle unverrechneten
  Konsultationen des aktuellen Falls und des aktuellen Mandanten eine Rechnung
  erstellen. Dies ist eine \glqq Abkürzung\grqq{} des normalen Wegs der
  Rechnungserstellung und eignet sich vor allem für Sofortrechnungen einzelner
  Konsultationen oder Leistungen.
\end{itemize}
\subsection{Fälle und Kons}
 %\usepackage{graphics} is needed for \includegraphics
\begin{wrapfigure}{l}{4cm}
  \includegraphics[width=4cm]{images/fallkonsview}
  \caption{Fälle und Kons}
  \label{fig:fallkons}
\end{wrapfigure}
Diese View (Abb. \ref{fig:fallkons} listet synoptisch Fälle und dazugehörige
Konsultationen (Nur Titel ohne Texte) auf. Wenn im oberen Bereich ein Fall
angeklickt wird, werden im unteren Bereich die zu diesem Fall gehörenden
Konsultationen angezeigt. Wenn eine Konsultation angeklickt wird, wird diese
Konsultation in der Konsultation-View (S. \ref{konsview} s. \pageref{konsview}) angezeigt.\\

\bigskip

\subsection{Konsultationen}
\label{view:konsultationen}
Dies ist eine Auflistung aller bisherigen Konsultationen des aktuell
selektierten Patienten, unabhängig vom jeweiligen Fall.
\index{Konsultationsliste} Zu jeder Konsultation
wird der Text ohne Formatierungen angezeigt. (S. Abb. \ref{fig:konslisteview}).\\
%\usepackage{graphics} is needed for \includegraphics
\begin{wrapfigure}{r}{7.5cm}
  \includegraphics{images/konslisteview}
  \caption{Konsultationsliste}
  \label{fig:konslisteview}
\end{wrapfigure}
Klick auf den (blauen) Titel einer Konsultation wählt diese Konsultation in der
Konsultation-View (s. S. \pageref{konsview}) aus.

Klick auf das Filter-Symbol rechts oben öffnet den Filter-Dialog (s. Abb. \ref{fig:konsfilter}).
\begin{figure}[htp]
\begin{center}
  \includegraphics{images/filterdialog}
  \caption{Filterdialog}
  \label{fig:konsfilter}
\end{center}
\end{figure}
\index{Konsultationsfilter} Hier können Sie bestimmte Kriterien eingeben, nach
denen die angezeigten Konsultationen gefiltert werden sollen (d.h. es werden nur
noch diejenigen Konsultationen angezeigt, die den Filterbedingungen
entsprechen).

Im Oberen Feld können Sie angeben, ob nur Konsultationen eines bestimmten Falls
oder aller Fälle angezeigt werden sollen. Im unteren Feld können Sie
Suchbegriffe eingeben, welche im Konsultationstext vorkommen müssen. Mehrere
Suchbegriffe können mit AND, OR, NOT, AND NOT und OR NOT miteinander verknüpft werden.
Beispielsweise findet \glqq Lorem AND NOT ipsum\grqq{} nur solche Konsultationen, deren
Text \glqq Lorem\grqq, nicht aber \glqq ipsum\grqq{}enthält.

Ganz unten können Sie schliesslich noch angeben, ob Gross/Kleinschreibung
beachtet werden soll, oder ob Suchbegriffe als reguläre Ausdrücke betrachtet
werden sollen. Eine genaue Erklärung dieses Themas würde hier zu weit führen;
Sie finden sehr viel Literatur dazu mit den Stichwörtern \glqq Regular
Expression\grqq{}oder \glqq Pattern Matching\grqq{}. Diese Technik erlaubt es,
den Suchbegriff mit verschiedensten Platzhaltern zu beschreiben. So würde z.B.
\glqq M[ae][iy]e?r\grqq{} nach allen Meiers, Mayrs etc. in allen Schreibweisen
suchen.


\subsection{Konsultation}
 \label{konsview}
Detailansicht eines Konsultationseintrags (S. Abb. \ref{fig:konsdetail}).


\begin{figure}[htp]
\begin{center}
  \includegraphics{images/konsview}
  \caption{Konsultation: Detail}
  \label{fig:konsdetail}
\end{center}
\end{figure}

\subsection{AUF}
Diese View dient der Festlegung einer Arbeitsunfähigkeit. (Abb \ref{fig:auf})
\index{AUF} \index{Arbeitsunfähigkeit}. %\usepackage{graphics} is needed for \includegraphics
\begin{figure}[htp]
\begin{center}
  \includegraphics{images/aufview}
  \caption{AUF-View}
  \label{fig:auf}
\end{center}
\end{figure}
Eine AUF bezieht sich immer auf einen bestimmten Fall. Wenn kein Fall markiert
ist, werden Sie aufgefordert, zunächst einen anzugeben.

Wenn Sie auf das \glqq Neu\grqq -Symbol (roter Stern) klicken, erscheint ein
Dialog, in dem Sie Anfang und Ende der neuen Arbeitsunfähigkeit festlegen
können. Klick auf das \glqq Drucker\grqq -Symbol öffnet eine Text-View, in der
Sie noch manuelle Änderungen am AUF-Text ergänzen können, bevor Sie das Zeugnis
definitiv ausdrucken oder aufs Fax senden.

\subsection{Rezepte}
In dieser View werden Rezepte aufgenommen. \index{Rezept}Klicken Sie auf das \glqq Neu\grqq
-Symbol (roter Stern) und es wird ein neues Rezept mit dem Aktuellen Datum
erstellt. Ziehen Sie Artikel mittels Drag\&Drop aus einer Artikelliste oder der
Dauermedikations-View in dieses Rezept. Mit Klick auf das \glqq Drucker\grqq -
Symbol öffnen Sie eine Text-View, in der Sie noch manuelle Änderungen anbringen
können, bevor Sie das Rezept definitiv auf den Drucker oder ein Faxgerät oder in
einen Export-Konnektor senden.
Hierzu muss eine Textvorlage namens \glqq Rezept\grqq
existieren, welche an einer Stelle den Platzhalter [Rezeptzeilen] enthält. Dort
werden die ausgewählten Artikel eingefügt.

\subsection{Heute}
Diese View dient dazu, Konsultationen eines bestimmten Zeitraums (standardmässig
des aktuellen Tages) darzustellen. Sie gibt einen Überblick über Verrechnung und
verrechnete Zeit jeder Konsultation einzeln und der Summe davon, Dies eignet
sich beispielsweise, um abends die Abrechnungen des Tages nachzukontrollieren.


\subsection{Diagnosen}


	\include{views/rest}
	
\chapter{Standardplugins}
	% *******************************************************************************
% * Copyright (c) 2007 by Elexis
% * All rights reserved. This document and the accompanying materials
% * are made available under the terms of the Eclipse Public License v1.0
% * which accompanies this distribution, and is available at
% * http://www.eclipse.org/legal/epl-v10.html
% *
% * Contributors:
% *    G. Weirich - initial implementation
% *
% *  $Id: einleitung.tex 2468 2007-06-02 17:21:49Z rgw_ch $
% *******************************************************************************
% !Mode:: "TeX:UTF-8" (encoding info for WinEdt)

\section{Einleitung}
Für das Erstellen von Briefen, Rezepten, Zeugnissen etc. greift Elexis
standardmässig auf ein vollwertiges Büroprogramm zurück: OpenOffice.

Dies muss zwar nicht zwingend so sein, da das Textsystem von Elexis als Plugin
ausgeführt wird. Man könnte also auch ein Plugin erstellen (lassen), das
Microsoft\texttrademark{}Office\texttrademark{} oder irgendein anderes Textsystem
benutzt. Wir möchten hier aber nur auf das standardmässig verwendete OpenOffice
eingehen.

OpenOffice.org geht auf das in den 80er Jahren entwickelte StarOffice zurück und
ist heute eine Office-Suite analog zu Microsoft Office, mit dem Unterschied,
dass es quelloffen und für mehrere Betriebssysteme erhältlich ist.

Zur Zusammenarbeit mit Elexis eignet sich OpenOffice ab Version 2.0, welches Sie
von \href{http://www.openoffice.org}{http://www.openoffice.org} für Ihr
Betriebssystem herunterladen und installieren können (Wenn Sie die Vollversion
von Elexis benutzen, ist OpenOffice bereits vorinstalliert).

Nach der Installation von OpenOffice und Elexis müssen Sie die beiden Programme
noch miteinander bekannt machen. Dies geschieht innerhalb von Elexis, indem Sie
das Textplugin so konfigurieren, dass es auf OpenOffice zugreift (Das Plugin
\glqq NOA-Text\footnote{Wir verwenden das Modul 'Nice Office Access' von \href{http://www.ubion.org}{www.ubion.org}
 zum Einbinden von OpenOffice}\grqq{}muss hierzu installiert sein, was standardmässig der Fall ist.

Wählen Sie im Menu \textsc{Datei - Einstellungen - Textverarbeitung}
Es erscheint eine Dialogbox wie in Abb \ref{fig:text1}. Wählen Sie dort \glqq
%\usepackage{graphics} is needed for \includegraphics
\begin{figure}[htp]
\begin{center}
  \includegraphics{images/text1}
  \caption{Konfiguration des Textplugins}
  \label{fig:text1}
\end{center}
\end{figure}

NOA-Text\grqq{}. Gehen Sie dann zum Feld OpenOffice.org (S. Abb. \ref{fig:text2}
und wählen Sie durch Klick auf \glqq definieren\grqq{}den Pfad zu Ihrer
OpenOffice-Installation aus.
%\usepackage{graphics} is needed for \includegraphics
\begin{figure}[htp]
\begin{center}
  \includegraphics{images/text2}
  \caption{Konfiguration der OpenOffice-Installation}
  \label{fig:text2}
\end{center}
\end{figure}
Ab dem, nächsten Start von Elexis sollte OpenOffice dann zur Verfügung stehen
(Bei der ersten Verwendung werden Sie noch den Lizenzbedingungen von
OpenOffice.org zustimmen müssen)




	\include{plugins/agenda}
	\include{plugins/notizen}
	\include{plugins/omnivore}
	% *******************************************************************************
% * Copyright (c) 2007 by Elexis
% * All rights reserved. This document and the accompanying materials
% * are made available under the terms of the Eclipse Public License v1.0
% * which accompanies this distribution, and is available at
% * http://www.eclipse.org/legal/epl-v10.html
% *
% *  $Id: icpc-plugin.tex 1951 2007-02-26 09:34:52Z rgw_ch $
% *******************************************************************************
% !Mode:: "TeX:UTF-8" (encoding info for WinEdt)


\section{Sgam-xChange}
Sgam-xChange ist ein Plugin, das den Austausch von beliebigen Daten zwischen elKG-Systemen ermöglicht. Sie können beispielsweise ein Laborblatt oder eine ganze KG aus Ihrem Programm exportieren, verschlüsselt zu einem Kollegen transportieren, der es dann in seine elKG importieren kann - sofern diese ebenfalls xChange-fähig ist.Sgam-xChange ist ein Plugin, das den Austausch von beliebigen Daten zwischen elKG-Systemen ermöglicht. Sie können beispielsweise ein Laborblatt oder eine ganze KG aus Ihrem Programm exportieren, verschlüsselt zu einem Kollegen transportieren, der es dann in seine elKG importieren kann - sofern diese ebenfalls xChange-fähig ist.

xChange ist ein offener Standard, der folgende Elemente beinhaltet:
\begin{itemize}
 \item Transport-Datenformat, damit die beteiligten elKG-Programme die Daten \textit{verstehen} können, die sie intern vermutlich auf ganz unterschiedliche Weise speichern.
\item Transport-Container, welcher eine Verschlüsselungs- und Transfertechnologie definiert, damit diese Daten sicher und vertraulich vom Sender zum Empfänger gelangen.
\end{itemize}
Um es jedem Hersteller von Praxisprogrammen zu ermöglichen, xChange zu implementieren, wurde der Standard bewusst auf die wesentlichsten Elemente limitiert, und er wird jedem Interessenten kostenlos zur Verfügung gestellt, inklusive einer Beispielanwendung, die konkrete Implementationshinweise gibt.

Sgam-xChange entstand im Auftrag und unter Sponsoring der Arbeitsgruppe \href{http://www.sgam.ch/informatics}{Sgam.informatics} , welche auch Ansprechpartner für alle Lizenzfragen ist.

\subsection{Voraussetzungen}


Eine Erläuterung und Beschreibung der technischen Hintergründe finden Sie hier .
 Dieser Artikel beschränkt sich im Folgenden auf die Beschreibung der in Elexis eingebauten Implementation von Sgam.xChange.

Um Daten aus Elexis mittels Sgam.xChange transportieren zu können, benötigen Sie ausser Elexis und dem Sgam.xChange-Plugin (welches standardmässig integriert ist), noch das Verschlüsselungsprogramm \href{http://www.gnupg.org}{GnuPG}.

 GnuPG ist für die meisten wichtigen Betriebssysteme (Windows, Linux, Mac) erhältlich und ist kostenlos. Es ist ein OpenSource-Verschlüsselungssystem, das seit 1999 existiert und weltweit einen ausgezeichneten Ruf geniesst (ist beispielsweise in manchen Ländern verboten, weil es von Geheimdiensten nicht geknackt werden kann). In der Schweiz ist seine Anwendung legal.

\subsection{Installation und Konfiguration}

GPG kann an beliebiger Stelle installiert werden; es muss dann lediglich an Elexis bekanntgemacht werden, wo es liegt. Öffnen Sie dazu im Menü \textit{Datei-Einstellungen} den Reiter \textit{xChange}:

\includegraphics[width=3in]{images/xc1.png}
% xc1.png: 590x350 pixel, 96dpi, 15.61x9.26 cm, bb=0 0 442 262

Folgende Einstellungen können hier vorgenommen werden:
\begin{itemize}
 \item GnuPG-Programm: Den vollständigen Pfad zu gpg.exe
\item Verzeichnis für Transferdaten: Ein existierendes Verzeichnis, in das xChange Dateien zwischenspeichern kann
\item Titel für Schlüsselanfragen: xChange kann automatische Mails generieren, um  den öffentlichen Schlüssel des Austauschpartners anzufordern. Hier können Sie den Titel einsetzen, den eine solche Mail haben soll.
\item Text für Schlüsselanfragen: Hierhin kommt der Text, welcher in eine Schlüsselanforderungs-Mail gesetzt werden soll.
\item Titel für Antworten: Auch für die Beantwortung einer Schlüsselanfrage kann xChange automatisch eine E-Mail generieren. Hierhin kommt der Titel einer solchen Antwort
\item Text für Antworten: Der Textinhalt einer Antwort-Mail
Signaturschlüssel (E-Mail): Hier sollten Sie denjenigen Schlüssel angeben, der standardmässig für xChange verwendet werden soll. Oft wird das sowieso Ihr einziger Schlüssel sein. Geben Sie dabei die E-Mail ein, die mit diesem Schlüssel Verknüpft ist.

\item
\item Neues Schlüsselpaar erstellen: Falls Sie noch keinen GPG-Schlüssel haben, können Sie nach Klick auf diesem Knopf ein Paar erstellen (Ein Schlüsselpaar besteht aus einem mit einem Passwort gesicherten privaten Schlüssel, den Sie zum Signieren und zum Entschlüsseln brauchen, und aus dem dazu passenden öffentlichen Schlüssel, den Sie jedem Interessenten geben können, um damit Nachrichten an Sie verschlüsseln und Signaturen von Ihnen verifizieren zu können)

\end{itemize}

Nach Knopfdruck erscheint dieses Fenster:

\includegraphics[width=3in]{images/xc2.png}
% xc2.png: 438x351 pixel, 96dpi, 11.59x9.29 cm, bb=0 0 328 263

Die Angaben sind selbsterklärend. xChange wird Ihre Sendungen jeweils mit diesem Schlüssel signieren und wird Sendungen, die an Sie gerichtet sind, mit diesem Schlüssel zu decodieren versuchen.
(Genauere Erklärungen zu den Schlüsseln finden Sie \href{http://www.elexis.ch/jp/index.php?option=content&task=view&id=64}{hier})
\subsection{Eine Krankengeschichte exportieren}
Klicken Sie mit der rechten Maustaste auf den gewünschten Patienteneintrag und wählen Sie \textit{KG Exportieren}

\includegraphics[width=3in]{images/xc3.png}
% xc3.png: 284x436 pixel, 96dpi, 7.51x11.53 cm, bb=0 0 213 327

 Es erscheint folgender Dialog:

\includegraphics[width=3in]{images/xc4.png}
% xc4.png: 438x244 pixel, 96dpi, 11.59x6.46 cm, bb=0 0 328 183

Nach Klick auf \textit{OK} wird die KG automatisch verschlüsselt und an den Empfänger gesandt, der sie nur mit seinem, oben unter \textit{Empfänger} genannten Schlüssel wieder entschlüsseln kann.

\subsection{Die Empfängerseite}
Auf der Empfängerseite kommt nun die eben abgeschickte E-Mail an:

\includegraphics[width=3in]{images/import1.png}
% import1.png: 644x369 pixel, 72dpi, 22.72x13.02 cm, bb=0 0 644 369

Der Mail-Anhang ist die xChange-Datei.

\includegraphics[width=3in]{images/import2.png}
% import2.png: 540x134 pixel, 72dpi, 19.05x4.73 cm, bb=0 0 540 134

Diese kann gespeichert und direkt nach Elexis eingelesen werden aus dem Menu Datei-Import.

\includegraphics[width=3in]{images/import3.png}
% import3.png: 613x399 pixel, 72dpi, 21.63x14.08 cm, bb=0 0 613 399



Sofern Sie im Besitz des korrekten Schlüssels sind, wird die KG jetzt in Ihrem Programm importiert.

Zum Konzept der Schlüsselverwaltung lesen Sie bitte \href{http://www.elexis.ch/jp/index.php?option=content&task=view&id=64}{hier} mehr Einzelheiten.


\section{Elexis-Iatrix}
Alternativer KG-Stil mit problemorientiertem Ansatz. Dieses Plugin ist Teil von Iatrix.

 Die Iatrix-Entwicklung wird von Dr. med. P. Schönbucher, Luzern finanziert, der es allen Elexis-Anwendern kostenlos zur Verfügung stellt. Hauptentwickler ist \href{http://www.elexis.ch/jp/component/option,com_contact/task,view/contact_id,2/Itemid,32/}{Daniel Lutz}. Anfragen zu diesem Plugin deshalb bitte an ihn.

\section{Elexis-externe-Dokumente}
Einbindung von Dokumenten, die in einem bestimmten Verzeichnis gespeichert sind. Dieses Plugin ist Teil von \href{http://www.iatrix.org}{Iatrix}.

\section{Elexis-ebanking-Schweiz}
Einbindung des Schweizer ESR/DTA-Systems zum bargeldlosen Zahlungsverkehr. Dieses Plugin ist Teil der Standard-Distribution. Es hat selbst keine Benutzerfunktionen, muss aber vorhanden sein, wenn andere Plugins diese E-Banking-Funktionen nutzen wollen.

%\section{Elexis Laborimport Viollier}


	\include{plugins/trustx}
	\include{plugins/privatnotizen}
	% !Mode:: "TeX:UTF-8" (encoding info for WinEdt)
\section{Tessinercode}
Einbindung von ICD-10 und Tessiner Code als Konsultationsdiagnosenschlüssel. Dieses Plugin ist Teil der Standard-Distribution. 
	\include{plugins/bildanzeige}
	\include{plugins/email}
	\include{plugins/aerztekasseimporter}
	% !Mode:: "TeX:UTF-8" (encoding info for WinEdt)
\section{Elexis-Befunde}
Einbindung textorientierter datierter Befundserien (z.B. Gewicht, BZ, Quick etc.). Dieses Plugin ist Teil der Standard-Distribution.
\subsection{Konfiguration}

Wenn das Plugin installiert ist, finden Sie im Menu Datei-Einstellungen eine Rubrik \textit{Befunde}. Diese wird anfangs leer sein (Abb. \ref{fig:befundesettings}):
\begin{figure}[htp]
    \includegraphics[width=4in]{images/befunde1.png}
    % befunde1.png: 538x515 pixel, 96dpi, 14.23x13.62 cm, bb=0 0 403 386
    \caption{Befunde-Einstellungen}\label{fig:befundesettings}
\end{figure}


Um einen neuen Befundparameter hinzuzufügen, klicken Sie auf  \textit{Hinzufügen}. Sie werden nach dem Namen dieses Parameters gefragt, wir wählen  Röntgen. Danach erscheint eine Karteikarte mit diesem Parameter, die wir jetzt noch mit den einzutragenden Datenspalten versehen müssen (Abb: \ref{fig:befunde2}).
\begin{figure}[htp]
\includegraphics[width=3in]{images/befunde2.png}
% befunde2.png: 538x515 pixel, 96dpi, 14.23x13.62 cm, bb=0 0 403 386
    \caption{Neuer Parameter}\label{fig:befunde2}
\end{figure}
Klicken Sie nach jeder Zeile auf  \textit{Apply}  bzw.  \textit{Anwenden}:
\begin{figure}[htp]
    \includegraphics[width=3in]{images/befunde3.png}
    % befunde3.png: 580x581 pixel, 96dpi, 15.34x15.37 cm, bb=0 0 435 436
        \caption{Neuer Parameter 2}\label{fig:befunde3}
\end{figure}

Wenn ein Feld mehrzeilig sein soll, klicken Sie die entsprechende Checkbox an. Eine Variante mit mehr als zwei Spalten sehen Sie in Abb. \ref{fig:befunde4}.:
\begin{figure}[htp]
    \includegraphics[width=4in]{images/befunde7.png}
    % befunde7.png: 580x520 pixel, 96dpi, 15.34x13.76 cm, bb=0 0 435 390
    \caption{Mehrspaltiger Parameter}\label{fig:befunde4}
\end{figure}

\pagebreak

\subsection{Anwendung}

Öffnen Sie die  Befunde-View.
\begin{flushleft}
\includegraphics[width=3in]{images/befunde4.png}
% befunde4.png: 276x394 pixel, 96dpi, 7.30x10.42 cm, bb=0 0 207 295
\end{flushleft}

Sie sehen dann die konfigurierten Messparameter:
\begin{flushleft}
\includegraphics[width=4in]{images/befunde5.png}
% befunde5.png: 621x636 pixel, 96dpi, 16.43x16.83 cm, bb=0 0 466 477
\end{flushleft}
Um eine neue Messung einzugeben, klicken Sie auf das grüne Pluszeichen rechts oben.
\begin{flushleft}
\includegraphics[width=3in]{images/befunde6.png}
% befunde6.png: 438x290 pixel, 96dpi, 11.59x7.67 cm, bb=0 0 328 217
\end{flushleft}
Sie sehen jetzt Ihre bei der Konfiguration angegebenen Messzeilen, ein- oder mehrzeilig. Mit Klick auf OK wird der neue Eintrag übernommen. Mit Doppelklick können Sie ihn wieder öffnen.

	\include{plugins/arzttarife}
	\include{plugins/artikel}
	
\chapter{Das Textsystem}
	% *******************************************************************************
% * Copyright (c) 2007 by Elexis
% * All rights reserved. This document and the accompanying materials
% * are made available under the terms of the Eclipse Public License v1.0
% * which accompanies this distribution, and is available at
% * http://www.eclipse.org/legal/epl-v10.html
% *
% * Contributors:
% *    G. Weirich - initial implementation
% *
% *  $Id: einleitung.tex 2468 2007-06-02 17:21:49Z rgw_ch $
% *******************************************************************************
% !Mode:: "TeX:UTF-8" (encoding info for WinEdt)

\section{Einleitung}
Für das Erstellen von Briefen, Rezepten, Zeugnissen etc. greift Elexis
standardmässig auf ein vollwertiges Büroprogramm zurück: OpenOffice.

Dies muss zwar nicht zwingend so sein, da das Textsystem von Elexis als Plugin
ausgeführt wird. Man könnte also auch ein Plugin erstellen (lassen), das
Microsoft\texttrademark{}Office\texttrademark{} oder irgendein anderes Textsystem
benutzt. Wir möchten hier aber nur auf das standardmässig verwendete OpenOffice
eingehen.

OpenOffice.org geht auf das in den 80er Jahren entwickelte StarOffice zurück und
ist heute eine Office-Suite analog zu Microsoft Office, mit dem Unterschied,
dass es quelloffen und für mehrere Betriebssysteme erhältlich ist.

Zur Zusammenarbeit mit Elexis eignet sich OpenOffice ab Version 2.0, welches Sie
von \href{http://www.openoffice.org}{http://www.openoffice.org} für Ihr
Betriebssystem herunterladen und installieren können (Wenn Sie die Vollversion
von Elexis benutzen, ist OpenOffice bereits vorinstalliert).

Nach der Installation von OpenOffice und Elexis müssen Sie die beiden Programme
noch miteinander bekannt machen. Dies geschieht innerhalb von Elexis, indem Sie
das Textplugin so konfigurieren, dass es auf OpenOffice zugreift (Das Plugin
\glqq NOA-Text\footnote{Wir verwenden das Modul 'Nice Office Access' von \href{http://www.ubion.org}{www.ubion.org}
 zum Einbinden von OpenOffice}\grqq{}muss hierzu installiert sein, was standardmässig der Fall ist.

Wählen Sie im Menu \textsc{Datei - Einstellungen - Textverarbeitung}
Es erscheint eine Dialogbox wie in Abb \ref{fig:text1}. Wählen Sie dort \glqq
%\usepackage{graphics} is needed for \includegraphics
\begin{figure}[htp]
\begin{center}
  \includegraphics{images/text1}
  \caption{Konfiguration des Textplugins}
  \label{fig:text1}
\end{center}
\end{figure}

NOA-Text\grqq{}. Gehen Sie dann zum Feld OpenOffice.org (S. Abb. \ref{fig:text2}
und wählen Sie durch Klick auf \glqq definieren\grqq{}den Pfad zu Ihrer
OpenOffice-Installation aus.
%\usepackage{graphics} is needed for \includegraphics
\begin{figure}[htp]
\begin{center}
  \includegraphics{images/text2}
  \caption{Konfiguration der OpenOffice-Installation}
  \label{fig:text2}
\end{center}
\end{figure}
Ab dem, nächsten Start von Elexis sollte OpenOffice dann zur Verfügung stehen
(Bei der ersten Verwendung werden Sie noch den Lizenzbedingungen von
OpenOffice.org zustimmen müssen)




	\include{text/vorlagen}
	% *******************************************************************************
% * Copyright (c) 2007 by Elexis
% * All rights reserved. This document and the accompanying materials
% * are made available under the terms of the Eclipse Public License v1.0
% * which accompanies this distribution, and is available at
% * http://www.eclipse.org/legal/epl-v10.html
% *
% * Contributors:
% *    G. Weirich - initial implementation
% *
% *  $Id$
% *******************************************************************************
% !Mode:: "TeX:UTF-8" (encoding info for WinEdt)
% Dieses Dokument enthält die Dokumentation der Platzhalter für Datenfelder

\section{Platzhalter für Datentypen}
\label{Platzhalter}
Diese Platzhalter können beispielsweise in Textdokumentvorlagen eingesetzt
werden, in eckige Klammern gesetzt, also z.B.[Patient.Name].
Sie können auch in der View 'Datenansicht' als Datenquellen eingesetzt werden.
Die folgende Liste erhebt keinen Anspruch auf Vollständigkeit; insbesondere
können durch Plugins zusätzliche Felder eingeführt werden.
\begin{description}
  \item [Anwender.Name] Name des aktuell eingeloggten Anwenders
  \item [Anwender.Vorname] Vorname des aktuell eingeloggten Anwenders
  \item [Anwender.Titel] Titel des aktuell eingeloggten Anwenders
  \item [Anwender.Kuerzel] Initialen des aktuell eingeloggten Anwenders
  \item [Anwender.Label] Login-Name des aktuell eingeloggten Anwenders
  \item [Mandant.Name,Vorname,Titel,Kuerzel,Label] dieselben Felder, wie bei Anwender, bezogen auf den aktuell
  aktiven Mandanten. 	
  \item [Mandant.EAN] Die EAN des aktuell aktiven Mandanten. Nur vorhanden, wenn
  das Plugin Arzttarife Schweiz geladen ist.
  \item [Mandant.KSK] Die KSK (bzw. ZSR)-Nummer des aktuell aktiven Mandanten.
  Nur vorhanden, wenn das Plugin Arzttarife Schweiz geladen ist.
  \item [Patient.Name,Vorname,Titel] Name etc.des aktuell selektierten Patienten
  \item [Patient.Geburtsdatum] Geburtsdatum des aktuell selektierten Patienten
  \item [Patient.Diagnosen] Diagnosen wie auf dem Titelblatt genannt
  \item [Patient.Allergien] Allergien wie auf dem Titelblatt
  \item [Patient.Strasse, Patient.Plz, Patient.Ort] Adresse des aktuell
  selektierten Patienten.
  \item [Patient.PersAnamnese] Anamnese wie auf dem Titelblatt
  \item [Patient.Telefon1, Patient.Telefon2, Patient.Natel] Telefonnummern
  \item [Patient.Medikation] Aktuelle Fixmedikation des aktuell selektierten
  Patienten
  \item [AUF.von] Beginn der aktuell ausgewählten Arbeitsunfähigkeit
  \item [AUF.bis] Ende der aktuell ausgewählten Arbeitsunfähigkeit
  \item [AUF.Prozent] Prozentsatz der aktuell ausgewählten AUF
  \item [AUF.Grund] Grund der aktuellen AUF (Unfall Krankheit)
  \item [AUF.Zusatz] Allfälliger Zusatztext
  \item [Fall.ArbeitgeberName] Name des Arbeitgebers, wenn eingetragen
  \item [Fall.Kostentraeger] Bezeichnung des Kostenträgers
  \item [Fall.KostentraegerKuerzel] Kürzel des Kostenträgers
  \item [Konsultation.Datum] Datum der aktuell ausgewählten Konsultation
  \item [Konsultation.Eintrag] Text der aktuell ausgewählten Konsultation
  \item [Konsultation.Diagnose] Diagnosen der aktuell ausgewählten Konsultation

\end{description}




\chapter{Mehrmandantenbetrieb}
	\include{multiuser}
	
\part{Mitmachen}
\include{mitmachen}

%%%%%%%%%%%%%%%%%%%%%%%%%%%%%%%%%%%%%%%%%%%%%%%%%%%%%%%%%%%%%%%%%%%%%%%%%%%%%%
\part{Anhänge}
\appendix

\chapter{Systemvoraussetzungen}
	% *******************************************************************************
% * Copyright (c) 2007 by Elexis
% * All rights reserved. This document and the accompanying materials
% * are made available under the terms of the Eclipse Public License v1.0
% * which accompanies this distribution, and is available at
% * http://www.eclipse.org/legal/epl-v10.html
% *
% *  $Id: voraussetzungen.tex 2472 2007-06-03 09:48:14Z rgw_ch $
%
%*******************************************************************************
% !Mode:: "TeX:UTF-8" (encoding info for WinEdt)

\section{Mindestanforderungen an die Hardware}
\label{systemvoraussetzungen}
\begin{itemize}
 \item Ein halbwegs aktueller PC (mind. 1 Ghz Taktfrequenz, mind. 512 MB RAM
 (empfohlen: 1 GB), mind. 1GB Harddiskplatz
\item  Eine Grafikkarte, die mindestens 1024x768 Pixel (empfohlen: 1280x1024)
anzeigen kann und ein dazu passender Monitor (z.B. 17-Zoll TFT oder grösser).
\item Empfohlen: Ein Drucker mit Schacht für A5-Papier (Rezepte, AUF-Zeugnisse) und ein oder zwei A4-Schächten (Für Briefe und Rechnungen)
\item Empfohlen: Ein Etikettendrucker.
\item Empfohlen: Ein externes Laufwerk zur Datensicherung.
\item Empfohlen: Von einer \textbf{Hardware}-Firewall geschützter Internet-Zugang (Personal Firewall ist ausdrücklich \textbf{nicht} empfohlen) (s. S. \pageref{sicherheit}).
\end{itemize}

\section{Unterstützte Betriebssysteme}
Elexis ist prinzipiell unter jedem Betriebssystem lauffähig, für das ein Java Runtime Environment Version 1.5
oder höher erhältlich ist. Im Speziellen sind das:
\begin{itemize}
\item  Windows 2000, XP \footnote{Windows Vista ist bisher nicht getestet.}
\item  Macintosh OS ab 10.4 (Tiger)  \footnote{Unter MacOS-X ist leider keine
OpenOffice-Einbindung möglich}
\item Linux (SuSE ab 9.3 oder Xubuntu/Kubuntu ab 6.06)\footnote{Achtung: Bei Linux mit Gnome-Desktop (wie Ubuntu) funktioniert die OpenOffice-Einbindung nicht, KDE (Kubuntu) und Xfce (Xubuntu) sind ok.}
\end{itemize}
Auf diesen Betriebssystemen sollte sich Elexis direkt mit einer der bereitgestellten Komplettversionen installieren lassen. Für andere Betriebssysteme kann mehr oder weniger Handarbeit notwendig sein.

Bitte beachten Sie, dass unsere Pauschalangebote und Wartungsverträge jeweils nur mit einem System möglich sind, das obengenannte Hard- und  Softwareanforderungen erfüllt.
\chapter{Demo in Vollprogramm umwandeln}
	% *******************************************************************************
% * Copyright (c) 2007 by Elexis
% * All rights reserved. This document and the accompanying materials
% * are made available under the terms of the Eclipse Public License v1.0
% * which accompanies this distribution, and is available at
% * http://www.eclipse.org/legal/epl-v10.html
% *
% * Contributors:
% *    G. Weirich
% *
% *  $Id: vollversion.tex 2469 2007-06-02 19:47:42Z rgw_ch $
% *******************************************************************************
% !Mode:: "TeX:UTF-8" (encoding info for WinEdt)

\label{vollversion}
\index{Vollversion}
Bei Elexis müssen Sie eigentlich nichts weiter tun, um die Vollversion zu
erhalten, wenn Ihnen die Demo gefallen hat: Die Demo-Version \textbf{ist} die
Vollversion. Der einzige Unterschied ist die Datenbank.

Um Die Demo in eine Vollversion umzuwandeln, sind folgende Schritte nötig:
\begin{itemize}
  \item \glqq Echte\grqq{}Datenbank-Engine installieren. (\ref{dbengine})
  \item Elexis mit dieser Datenbank-Engine verbinden (\ref{connect})
  \item Die Demo-Datenbank löschen.
  \item Aktuelle Versionen der Stammdaten einlesen
  \item Elexis-Grundkonfiguration für Ihre Praxis erstellen
\end{itemize}
Allerdings sind diese Schritte nicht ganz trivial. Das Aktualisieren des
Datenbestands und vor allem das Erstellen einer korrekten Grundkonfiguration für
Ihre Praxis kann recht aufwändig sein. Hier kann sich Elexis nicht von anderen
Praxisprogrammen unterscheiden, da die Komplexität der Aufgabe für alle dieselbe
ist.
Wenn Sie dies selber machen wollen, sollten Sie sich genügend Zeit nehmen (einen
ganzen Tag), und dieses Kapitel exakt durchgehen. Wenn Sie nicht sicher sind,
empfehlen wir Ihnen, die Installation und Basiskonfiguration, zusammen mt einer
Instruktionsstunde für Ihr Praxispersonal einzukaufen.

\section{Datenbank-Engine installieren}
\label{dbengine}
Eine Elexis-Installation besteht aus zwei Teilen: Einem sogenannten \textit{Server}, auf dem die Daten 
abgelegt sind, und einem oder mehreren \textit{Clients}, welche auf die Daten zugreifen und diese darstellen
und bearbeiten lassen. Server und Client können auf ein- und demselben oder auf verschiedenen Computern sein.
Ein \textit{\textit{Server}} in einem weiteren Sinn ist ein eigener Computer, auf dem ein oder mehrere
Server-Programme laufen.

Als Server-Programm verwendet Elexis eine (im Prinzip beliebige) Datenbank, welche sich nach dem
Industriestandard JDBC bedienen lässt. Die automatische Einrichtung ist vorkonfiguriert für folgende
Datenbanksysteme:
\begin{itemize}
\item MySQL (\href{http://www.mysql.com}{www.mysql.com}): Dies ist die verbreitetste Datenbank im Internet. Die allermeisten
datenbank-basierten
 Web-Anwendungen verwenden einen MySQL-Server im Hintergrund. MySQL gilt als schnell, ausgereift und ist
 unkompliziert zu installieren. Für kommerzielle Zwecke kostet ein MySQL-Server ca. Fr. 750.-. Für private
 Zwecke ist er kostenlos.

\item PostgreSQL (\href{http://www.postgresql.org}{www.postgresql.org}): Dies ist
ein OpenSource Datenbankserver. Er beherrscht einen grösseren Befehlssatz als MySQL,
 gilt aber als etwas langsamer als dieser. Für unsere Zwecke sollte dies aber keine Rolle spielen, da die
 Geschwindigkeitstests sich üblicherweise im Bereich von einigen tausend Zugriffen pro Sekunde bewegen;
 ein Wert, der wohl nur in den wenigsten Arztpraxen erreicht werden dürfte. PostgreSQL ist kostenlos für alle
 Zwecke.

\item HSQLDB: Dies ist eine in Java geschriebene OpenSource-Datenbank. Sie kann sowohl als separater Server,
 als auch im Programm integriert verwendet werden. HSQL ist etwas langsamer, als die beiden erstgenannten Systeme,
 für kleinere Umgebungen aber eventuell genügend. Allerdings muss man ein
 besonderes Augenmerkt auf die Datensicherung legen, da ein Computerabsturz
 (oder versehentliches Ausschalten des PC, ohne das Programm herunterzufahren) 
 die Datenbank unbrauchbar machen kann. HSQL ist kostenlos. 

\end{itemize}

Die Demo-Installation basiert auf einem InProc HSQL-Server. Falls Sie eine Client-Server Installation möchten
(also mehrere Clients, die auf einen gemeinsamen Datenbestand zugreifen, dann sollten Sie als erstes den Server 
installieren.
\begin{itemize}
 \item Wählen Sie am besten einen eigenen Computer als Server, an dem nicht direkt gearbeitet wird.
  Es dürfen aber ohne weiteres mehrere Serverprogramme darauf laufen (z.B.
  Mailserver, Fax-Server, Druck-Server etc.).
\item Installieren Sie dort den Datenbankserver Ihrer Wahl (Wir empfehlen mysql oder PostgeSQL)
\item Erstellen Sie auf der Datenbank einen User-Account mit dem Namen elexisuser
\item Erstellen Sie eine (leere) Datenbank mit dem Namen elexis, auf den 'elexisuser' Vollzugriff hat
\item Entscheiden Sie sich unbedingt für eine Backupstrategie. Mehr Informationen dazu weiter unten.
\item Die weitere Konfiguration erfolgt von den Clients aus. Der Server braucht im Alltagsbetrieb auch nicht 
unbedingt Bildschirm und Tastatur, sondern er kann an irgendeiner kühlen Stelle
der Praxis oder im Keller unauffällig aufgestellt werden.
\end{itemize}

\textbf{Wichtig!}

\textit{Vergessen Sie nicht die Datensicherung!}

Elexis speichert sämtliche Daten in dieser Datenbank. Eine Zerstörung der Datenbank ist keineswegs unmöglich.
Ein Stromausfall kann die Festplatte  \glqq  auf dem linken Fuss\grqq{}  erwischen,
ein mechanischer Schaden kann wichtige Sektoren der Platte vernichten und sie
unlesbar machen, ein Fehler eines Programms kann die Daten löschen, ein Virus
kann sich an Ihren Daten austoben. Es gibt viele Datensicherungsstrategien, 
einige wollen wir hier vorstellen:

\begin{description}
\item[ Replikation ] Einige Datenbanken, wie z.B. MySQL ab Version 4.0 können ihre Daten laufend auf einen
auf einem anderen Computer laufenden Server kopieren. Da jeweils nur die geänderten Daten im Hintergrund
übertragen werden, kostet das weniger Leistung, als man vielleicht zunächst denken würde. Dieses Verfahren 
nennt man \textit{Replikation} . Im Endeffekt hat man dadurch zwei Datenbanken mit identischem Inhalt. 
Wenn der Server kaputt geht, kann man innert weniger Minuten den zweiten Rechner zum Server ernennen und
praktisch ohne Unterbruch weiterarbeiten.
\item [Virtual Machine] Ein verwandtes Konzept: Man lässt den Datenbankserver auf einer hierfür reservierten
 virtuellen Maschine laufen (z.b. von \href{http://www.vmware.com/}{VMWare}) und sichert in regelmässigen
 Abständen die komplette virtuelle Maschine. Bei einem Serverausfall kann man ebenfalls innert Minuten die
 Sicherungs-VM auf demselben oder irgendeinem anderen Computer im Netz starten und weiterarbeiten.
\item [Häufige Datensicherung] Man kann ein automatisiertes Backup alle paar Minuten durchführen
 (z.B. mit mysqldump), und die so gesicherten Daten in mehreren Generationen auf verschiedenen Datenträgern
 aufbewahren. Diese Methode braucht die wenigesten Ressourcen von den hier genannten und erzeugt die kleinsten
 Backup-Dateien. Dafür braucht man mehr Zeit, um bei einem Serverausfall weiterarbeiten zu können: Man muss
 zuerst den Datenbankserver auf einem Reservecomputer starten und dann dort die gesicherten Daten einspielen,
 anschliessend je nach Konfiguration alle Clients auf den neuen Server einstellen.
\end{description}


\section{Verbindung mit der Datenbank herstellen}
\label{connect}
Starten Sie Elexis und wählen Sie im Menu Datei -> Verbindung.

\includegraphics[width=0.9\textwidth]{images/verbindung11.png}
% verbindung11.png: 1024x768 pixel, 72dpi, 36.12x27.09 cm, bb=0 0 1024 768



Geben Sie den Typ der Datenbank (hier mysql), die Adresse des Servers (hier 192.168.0.2) oder dessen Internet-Namen (z.B. testserver.elexis.ch) ein, sowie den Namen der Datenbank (hier  elexistest) ein und klicken Sie auf \textit{weiter}.

\includegraphics[width=0.9\textwidth]{images/verbindung12.png}
% verbindung12.png: 1024x768 pixel, 72dpi, 36.12x27.09 cm, bb=0 0 1024 768
Geben Sie hier in der oberen Zeile den Benutzernamen für die Datenbank (hier testuser) ein, in die untere Zeile das passende Passwort (hier testelexis) und klicken Sie auf \textit{fertig stellen}.

\includegraphics[width=0.9\textwidth]{images/verbindung13.png}
% verbindung13.png: 1024x768 pixel, 72dpi, 36.12x27.09 cm, bb=0 0 1024 768

 Es erscheinen einige Fehlermeldungen, die Sie wegklicken können. Danach müssen Sie das Programm nochmals neu starten.

\includegraphics[width=4in]{images/verbindung14.png}
% verbindung14.png: 1024x768 pixel, 72dpi, 36.12x27.09 cm, bb=0 0 1024 768

Sie können sich jetzt mit dem Namen \textit{Administrator} und dem Passwort \textit{admin} an ihrem eben erstellten Elexis-System anmelden.


\section{Demo-Datenbank löschen}
Wenn Elexis beim Start eine Demo-Datenbank vorfindet, wird es immer mit dieser
verbinden,egal was Sie für Verbindungseinstellungen definiert haben. Sie müssen
deshalb jetzt Elexis beenden und dann die Demo-Datenbank löschen oder
umbenennen. Diese ist im Elexis-Programmverzeichnis im Verzeichnis \glqq
demoDB\grqq{} zu finden. Löschen Sie dieses Unterverzeichnis oder benennen Sie
es um. Danach können Sie Elexis wieder starten, es sollte sich jetzt mit Ihrer
vorher neu eingerichteten Datenbank verbinden.

\section{Stammdaten einlesen}
\subsection{Tarmed}
\index{Tarmed}
\subsection{ICD-10}
\index{ICD-10}
\subsection{Analyseliste}
\index{Analysenliste}\index{AL}
\subsection{Spezialitätenliste}
\index{Spezialitätenliste}\index{SL}
\subsection{MiGeL}
\index{MiGeL}

\section{Grundkonfiguration}
\index{Grundkonfiguration}
\label{grundkonfiguration}
Die Grundkonfiguration umfasst folgende Schritte:
\begin{itemize}
  \item Mandanten und Anwender einrichten
  \item Laborparameter definieren
  \item Textvorlagen erstellen
\end{itemize}
\subsection{Mandanten und Anwender einrichten}
Öffnen Sie die Perspektive \textit{Kontakte},

\includegraphics[width=4.5in]{images/grundkonfkonta.png}
% grundkonfkonta.png: 1024x585 pixel, 72dpi, 36.12x20.64 cm, bb=0 0 1024 585
\begin{itemize}
 \item Geben Sie unter \textit{Bezeichnung1} den Namen des neuen Mandanten oder Anwenders ein und klicken Sie auf \textit{Neu erstellen}
 \item Suchen Sie dann den eben erstellten Eintrag in der oberen Liste wieder auf, klicken Sie ihn an und ergänzen Sie die Angaben in der unteren Hälfte. Wie immer bei Elexis brauchen Sie nicht unbedingt alle Felder auszufüllen. \textit{Ernennen} Sie den eben erstellten Kontakt dann zum \textit{Mandanten} oder \textit{Anwender} (Ein Mandant ist immer auch gleichzeitig ein Anwender, und beide sind immer auch \textit{Personen}
 \item Wenn Sie alle Mandanten und Anwender so eingegeben haben, gehen Sie auf Datei-Einstellungen und dort auf den Reiter \textit{Zugriffsteuerung - Mandanten}
\end{itemize}

\includegraphics[width=4in]{images/grundkonfmand.png}
% grundkonfmand.png: 580x549 pixel, 72dpi, 20.46x19.37 cm, bb=0 0 580 549
\begin{itemize}
 \item Geben Sie dort die entsprechenden Angaben für Ihre eingerichteten Mandanten ein. Für \textit{Kürzel} muss der Anmeldename angegeben werden, für Passwort das Anmeldepasswort. Die restlichen Daten hängen vom Typ des Mandanten ab.
 \item Gehen Sie dann zum Abschnitt Zugriffsteuerung - Anwender
\end{itemize}

Geben sie dort für alle definierten \index{Anwender}Anwender die entsprechenden Daten ein. Vergessen Sie auch nicht, allen Anwendern einen existierenden Standard-Mandanten zuzuordnen (\textit{Für Mandant}). Auch für bereits angelegte Mandanten-Einträge (die Sie auch in der Anwender-Einstellung wiederfinden, da alle Mandanten auch Anwender sind), sollte ein Standard-Mandant festgelegt werden (Normalerweise er selbst). Der Zugeordnete Mandant definiert, in wessen Auftrag und auf wessen Rechnung der betreffende Anwender normalerweise arbeitet. Dies kann während der Arbeit geändert werden (Unter \textit{Datei - Mandant}...), aber beim Einloggen wird immer zunächst der Standard-Mandant eingestellt.

\subsection{Laborparameter eingeben}
Öffnen Sie zunächst wieder die \textit{Kontakte-View}. Geben Sie dort Ihr Eigenlabor und ihre externen Labors ein. Markieren Sie diese als \textit{Labor}

\includegraphics[width=4in]{images/grundkonfmand1.png}
% grundkonfmand1.png: 752x585 pixel, 72dpi, 26.53x20.64 cm, bb=0 0 752 585
\subsection{Textprogramm konfigurieren}

Elexis arbeitet bisher nur mit OpenOffice zusammen. Deshalb wird hier auch nur die Konfiguration mit OpenOffice erläutert.

\begin{itemize}
 \item Falls noch nicht geschehen: Installieren Sie OpenOffice (Mindestens Version 2.0)
 \item Wählen Sie in Elexis \textit{Datei-Einstellungen-Textverarbeitung}
\end{itemize}

\includegraphics[width=3.5in]{images/grundkonfmand2.png}
% grundkonfmand2.png: 580x480 pixel, 72dpi, 20.46x16.93 cm, bb=0 0 580 480

\begin{itemize}
 \item Suchen Sie den Pfad des \textit{program} Unterverzeichnisses der OpenOffice Installation auf. Unter Windows wird das standardmässig in c:/programme/OpenOffice.org 2.0/program sein, bzw. an der Stelle, wo Sie OpenOffice installiert haben.
 \item Drücken Sie \textit{Apply}, Schliessen Sie die Konfiguration und starten Sie Elexis neu
 \item Wenn Sie jetzt z.B, die Briefe-Perspektive anwählen, sollte das Fenster von OpenOffice innerhalb des Elexis-Fensters erscheinen. (Dies wird beim ersten Mal recht lange dauern, d.h. ca. 30 Sekunden)
\end{itemize}

\subsubsection{\index{Vorlagen!Druckvorlagen}Druckvorlagen erstellen}
Für einige Formulare sucht Elexis nach vordefinierten Druckvorlagen mit einem festgelegten Namen. Diese definieren das für Ihre Anwendung spezifische Aussehen dieser Formulare. Es müssen jeweils Platzhalter für Variable Daten an den passenden Stellen eingefügt werden.

Um eine Formatvorlage zu erstellen, gehen Sie am besten so vor: Schreiben Sie die Vorlage ganz normal in der Textverarbeitung und speichern Sie sie als normales Textdokument. Von Elexis aus wählen Sie dann die Perspektive \textit{Briefe} und rufen das

\includegraphics[width=2.5in]{images/import.png}
% import.png: 297x226 pixel, 96dpi, 7.86x5.98 cm, bb=0 0 223 169

ViewMenu rechts auf.

Wählen Sie dort \textit{Text importieren} und suchen Sie ihre vorher erstellte Vorlage auf.
 Hierdurch wird das Dokument nach Elexis importiert. Sie können jetzt noch Änderungen anbringen, rufen Sie dann wieder das ViewMenu rechts auf und wählen Sie diesmal \textit{Als Vorlage speichern}.

\includegraphics[width=2.5in]{images/rezept1.png}
% rezept1.png: 219x147 pixel, 72dpi, 7.73x5.19 cm, bb=0 0 219 147

Für \textit{Name der Vorlage} müssen Sie bei den unten aufgelisteten Standardvorlagen den entsprechenden Namen geben, für eigene Vorlagen können Sie beliebige Bezeichnungen einsetzen. Unter \textit{Mandant} können Sie einstellen, für welchen Mandanten diese Vorlage ist (oder ob für alle)

Im Folgenden jetzt eine Liste der erwarteten Standard-Vorlagen:

\begin{itemize}
\item Rezept: \index{Vorlagen!Rezept}Hierfür wird eine Formatvorlage mit dem Namen \textit{Rezept} benötigt. Diese kann z.B. so aussehen:
 \end{itemize}
\includegraphics[width=4in]{images/rezept.png}
% rezept.png: 477x290 pixel, 72dpi, 16.83x10.23 cm, bb=0 0 477 290

An der Stelle, wo [Rezeptzeilen] steht, werden später die ausgewählten Medikamente eingesetzt. Dieser Platzhalter ist somit notwendig. Alle anderen Elemente der Vorlage \textit{Rezept} sind fakultativ.

\begin{itemize}
 \item \textit{AUF-Zeugnis}: Eine Formatvorlage für Arbeitszeugnisse. Als Platzhalter können [AUF.Grund], [AUF.von], [AUF.bis], [AUF.Prozent] und alle Standard-Platzhalter eingesetzt werden. Alle sind fakultativ.


 \item \textit{Laborblatt}: Für das Ausdrucken von Laborwerten. Die Laborwerte werden beim Platzhalter [Laborwerte] eingesetzt, dieser Platzhalter ist zwingend, andere können nach Wahl gesetzt werden.
\end{itemize}

	%%%%%%%%%%%%%%%%%%%%%%%%%%%%%%%%%%%%%%%%%%%%%%%%%%%%%%%%%%%%%%%%%%%%%%%%%%%%%%

	%\include{installs/serverinst}
	%\include{installs/windowsinst}
	%\include{installs/linuxinst}
	%\include{installs/macinst}
	%\include{installs/dbanbindung}
	%\label{installs/server}
	%\include{installs/server}


%%%%%%%%%%%%%%%%%%%%%%%%%%%%%%%%%%%%%%%%%%%%%%%%%%%%%%%%%%%%%%%%%%%%%%%%%%%
\chapter{Zusatzplugins}
    % *******************************************************************************
% * Copyright (c) 2007 by Elexis
% * All rights reserved. This document and the accompanying materials
% * are made available under the terms of the Eclipse Public License v1.0
% * which accompanies this distribution, and is available at
% * http://www.eclipse.org/legal/epl-v10.html
% *
% *  $Id: extplugins.tex 2472 2007-06-03 09:48:14Z rgw_ch $
% *******************************************************************************
% !Mode:: "TeX:UTF-8" (encoding info for WinEdt)

Dieser Anhang listet einige bekannte Zusatzplugins für Elexis auf. Es muss nochmals betont werden, dass niemals alle existierenden Plugins bekannt sein können, und dass es darum auch keine abschliessende Liste geben kann. Wir möchten aber unabhängige Plugin-Entwickler ermutigen, ihre Plugins zu melden, damit wir sie in späteren Auflagen dieses Handbuchs mit aufführen können.
Wenden Sie sich bitte für weitere Informationen zu diesen Plugins an die bei jedem Plugin genannte Kontaktadresse. Bezugs- und Nutzungsbedingungen für externe Plugins können anderes sein, als für das Hauptprogramm (Beispielsweise können externe Plugins kostenpflichtig sein, und externe Plugins müssen auch nicht unbedingt quelloffen sein).

\section{ICPC-2}
Dieses Plugin implementiert den International Code for Primary Care in Elexis. ICPC-2 ist ein modernes, Praxisnahes Klassifikationsmodell für Konsultationsgründe, Diagnosen und Massnahmen. S, z.B. folgenden Link:
\href{http://www.primary-care.ch/pdf/2005/2005-10/2005-10-656.PDF}{http://www.primary-care.ch/pdf/2005/2005-10/2005-10-656.PDF}


ICPC-2 ist speziell auch für Statistik und hausärztliche Forschung geeignet. Allerdings ist ICPC-2 leider nicht frei verfügbar, sondern mit Lizenzgebühren belastet. Daher enthält dieses Plugin auch nicht den Code, sondern nur die Umgebung, in die der Code eingelesen und angewendet werden kann. Für die Lizensierung in der Schweiz ist die Schweizerische Gesellschaft für Allgemeinmedizin zuständig (\href{http://www.sgam.ch}{www.sgam.ch})

\section{Elexis Laborimport Risch}

Import von Laborwerten (und Parametern) des \href{http://www.risch.ch}{Labors Risch}.

\section{Elexis Laborimport Krech}

Import von Laborresultaten(und Parametern) aus den \href{http://www.labor.ch}{Labors Krech} , Kreuzlingen und MicroGen (und \textit{kompatible}).

\section{Elexis-Importer-PraxisDesktop}

Import von Patientendaten aus dem Programm \href{http://www.praxisdesktop.ch}{PraxisDesktop}. Dieses Plugin wird separat vertrieben.


 \section{Elexis-Importer-Vitomed}

 Import von Stammdaten des Programms Vitomed (\href{http://www.vitomed.ch}{www.vitomed.ch}


\section{Buchhaltung}
Einfache Praxisbuchhaltung. Dieses Plugin wird separat von Elexis vertrieben und ist kostenpflichtig. 
\chapter{Überlegungen zur Datensicherheit}	
	\include{appendix/sicherheit}

\chapter{Index}
\printindex
\end{document}
