% *******************************************************************************
% * Copyright (c) 2007 by Elexis
% * All rights reserved. This document and the accompanying materials
% * are made available under the terms of the Eclipse Public License v1.0
% * which accompanies this distribution, and is available at
% * http://www.eclipse.org/legal/epl-v10.html
% *
% * Contributors:
% *    G. Weirich, U. Berner, D. Lutz
% *
% *  $Id: elexis.tex 2967 2007-08-07 14:16:52Z rgw_ch $
% *******************************************************************************
% !Mode:: "TeX:UTF-8" (encoding info for WinEdt)
%
% Dies ist das Zentraldokument für die Elexis-Dokumentation. Einzelne
% Kapitel und Unterkapitel können mit \include eingesetzt werden.

\documentclass[paper=a4,BCOR8.25mm,twoside]{scrartcl}
\usepackage{german}
\usepackage[utf8]{inputenc}
\usepackage{makeidx}
\usepackage{wrapfig}
\makeindex
% Hier ein etwas skurriler Block, der dazu dient, die Unterschiede
% zwischen PDFLaTeX und latex auszubügeln
% Grafiken müssen als png oder gif (für PDFLaTeX) und als eps (für Latex)
% vorhanden sein. Die Endung kann man beim \includegraphics jeweils weglassen,
% das System nimmt je nach Renderer die geeignete Variante.

\newif\ifpdf
\ifx\pdfoutput\undefined
	\pdffalse              	%normales LaTeX wird ausgeführt
\else
	\pdfoutput=1
	\pdftrue               	%PDFLaTeX wird ausgeführt
\fi

\ifpdf
	\usepackage[pdftex]{graphicx}
	\DeclareGraphicsExtensions{.pdf,.jpg,.png}
\else
	\usepackage[dvips]{graphicx}
	\DeclareGraphicsExtensions{.eps}
\fi

\usepackage{floatflt}
\usepackage[]{hyperref}
\usepackage{color}
\author{Gerry Weirich}

\extratitle{
    \vfill
	\begin{center}
		\includegraphics{../elexis/rsc/elexis-logo}
	\end{center}
    \begin{center}
        \textbf{Version 1.1}
    \end{center}
    \vfill
}
\title{Die technische Dokumentation zu Elexis\textsuperscript{\textregistered}}
\begin{document}

\maketitle

\tableofcontents
\part{Einführung}

Diese technische Dokumentation ist als Referenz für Dienstleister gedacht, welche Installation und Support für Elexis\textsuperscript{\textregistered} anbieten wollen. Diese Dokumentation ist \textsuperscript{\textcopyright} 2007 by G.Weirich. Nachdruck, auch auszugsweise ist nur mit schriftlicher Genehmigung erlaubt.

\bigskip

Die Kenntnis des Elexis-Benutzerhandbuchs wird in dieser Referenz vorausgesetzt. Ebenso werden Grundkenntnisse über die Konfiguration und Bedienung der Betriebssysteme Windows 2000/XP/Vista, Linux KDE/Gnome und MacOS-X 10.x vorausgesetzt. Programmierkennntisse sind hingegen nicht notwendig.

\bigskip

Um eine gleichbleibend hohe Qualität aller Elexis-Installationen und eine maximale Kundenzufriedenheit zu gewährleisten, sind die in diesem Dokument festgelegten Richtlinien für die Zertifizierung als Elexis-Experte als bindend zu betrachten.

\part{Die Installation beim Kunden}
\section{Planung und Ablauf}
Wir empfehlen für die Installation folgendes Vorgehen:
\begin{itemize}
    \item Überprüfung der technischen Voraussetzungen (\ref{voraussetzungen}): Wenn die Hardware des Kunden diesen Anforderungen nicht genügt, kann ein Komplettpaket bestehend aus Hardware und Software kalkuliert und offeriert werden.
    \item Besprechung der Kundenspezifischen fragen (\ref{besprechung})
    \item Durchführung der Installation (\label{ablauf})
    \item Grundkonfiguration (\label{config})
    \item Erstinstruktion
    \item Folgeinstruktion nach ein bis zwei Wochen
\end{itemize}

Der Zeitaufwand für Installation und Konfiguration ist mit 3-4 Stunden zu kalkulieren (wird mit zunehmender Erfahrung weniger, letztlich ca. 2 Stunden). Basis-Instruktion und Folgeinstruktion sind mit je einer Stunde zu kalkulieren.

Wir empfehlen dem Kunden stets, jeweils das ganze Paket inklusive Folgeinstruktion zu kaufen, da die Erfahrung zeigt, dass die Anwender oft nach den ersten paar Tagen noch wesentliche Fragen haben. Es können aber auch je nach Wünschen und Möglichkeiten des Dienstleisters und des Kunden andere Installations- und Instruktionsvereinbarungen abgeschlossen werden.

\section{Richtlinien: Technische Voraussetzungen}
\label{voraussetzungen}
\subsection{Client}
Elexis benötigt für die Client-Seite ein System, das mindestens folgende Voraussetzungen erfüllt:
\begin{itemize}
    \item Bildschirm und Graphikkarte, die mindestens SXGA (1280x1024 Pixel) Darstellen können
    \item Prozessor mindestens der 1.5 GHz Klasse
    \item Speicheranforderung je nach Betriebssystem:
    \begin{itemize}
        \item Windows 2000:     mindestens 512MB
        \item Windows XP:       mindestens 768MB
        \item Windows Vista:    mindestens 1 GB
        \item Linux mit KDE oder Gnome: mindestens 512MB
        \item Linux mit XFCE:   mindestens 384MB
        \item MacOS-X 10.4:     mindestens 1GB
    \end{itemize}
    \item Festplattenplatz: mindestens 1GB frei
\end{itemize}
Die Installation auf einem System, welches diese Erfordernisse nicht erfüllt, sollte strikt abgelehnt werden. Andernfalls ist mit schwierig zu behandelnden Supportanfragen zu rechnen, manchmal kommt es bei \glqq zu klein\grqq ausgelegten Systemen auch je nach parallel eingesetzter Fremdsoftware zu sporadisch auftretenden Fehlern, welche nur schwer identifiziert werden können.
Ein PC, welcher Daten von Laborgeräten einlesen soll, sollte eine serielle Schnittstelle haben, da viele Laborgeräte ihre Werte so liefern.

\bigskip
Elexis benötigt ausserdem ein Java-Runtime Environment Version 5.0 und OpenOffice Version 2.03. Beides ist im Basis-Installationspaket enthalten. Wir empfehlen keine separate Installation dieser Komponenten, weil sonst die Gefahr besteht, dass der Kunde -willentlich oder unwillentlich- ein Update oder sonst eine Veränderung einer der Komponenten durchführt, welche zu Inkompatibilitäten führt. Wir empfehlen, Elexis und das gebundelte jre und ooo als untrennbare Einheit zu behandeln.

\subsection{Server}
Die Anforderungen an den Server hängen naturgemäss von der Zahl der Clients und deren Aktivität ab. Als Faustregel wird für eine Praxis mit bis zu 10 Arbeitsplätzen ein Gerät genügen welches folgende Anforderungen erfüllt:
\begin{itemize}
    \item Prozessor mindestens der 1.5 GHz Klasse, idealerweise DualCore
    \item Mindestens 1GB RAM
    \item Mindestens 20GB Harddisk-Kapazität, SATA RAID 1 oder RAID 5
    \item Betriebssystem Linux ohne grafische Oberfläche.
\end{itemize}

Für eine Praxis mit bis zu 50 Arbeitsplätzen ist mindestens ein Gerät erforderlich, welches folgende Anforderungen erfüllt:
\begin{itemize}
    \item Zwei DualCore oder QuadCore-Prozessoren
    \item Mindestens 4GB RAM
    \item Mindestens 80GB Harddisk-Kapazität, SATA organisiert als RAID 5
    \item Betriebssystem Linux ohne grafische Oberfläche
\end{itemize}

Eine Installation mit mehr als 50 Arbeitsplätzen wird von uns offiziell nicht unterstützt.

\subsubsection{Peer to Peer}
Ein Peer2Peer Netz sollte nicht aktiv beworben und empfohlen werden. Es darf ausnahmsweise eingesetzt werden, wenn der Kunde ausdrücklich nach einer Lösung ohne separaten Server fragt, und wenn die Gesamtinstallation nicht mehr als drei Arbeitsplätze umfasst. Das Server-Programm darf dann auf dem am wenigsten belasteten Arbeitsplatz installiert werden. Dieser muss dann aber zwingend mindestens 1 GB RAM und mindestens einen Dual Core Prozessor haben.

\subsubsection{Server-Programm}
Die meisten Erfahrungen liegen mit MySQL als Server-Programm vor. Deshalb empfehlen wir dieses als erste Wahl. Allerdings ist zu beachten, dass die MySQL-Lizenz in kommerziellem Umfeld kostenpflichtig ist.

Als zweite Wahl empfehlen wir PostgreSQL. Für kleinere Netze sind die beidem Empfehlungen als gleichwertig zu betrachten, bei grösseren Netzen (ab 5 Arbeitsplätzen) würden wir MySQL den Vorzug geben. Dies ist aber keine bindende Empfehlung. Verwenden Sie den Server, mit dem Sie selber die meiste Erfahrung haben.

Andere Server-Programme werden von uns nicht unterstützt.
Insbesondere ist es strikt abzulehnen, den für die DemoDB verwendeten HSQL-Server in der produktiven Umgebung weiterzuverwenden (Bereits ein Stromausfall, Fehlbedienung oder Computerabsturz kann die hsql-Datenbank unrettbar zerstören).

\subsection{Drucker}
Wir empfehlen generell einen Drucker mit mindestens zwei Schächten. Der Drucker sollte PCL5 oder 6 und/oder PostScript beherrschen, und er sollte über die parallele Schnittstelle oder via LAN angeschlossen sein. GDI-Drucker sind strikt abzulehnen. Drucker, die via USB angeschlossen werden, können beim Rechnungsdruck Probleme verursachen.

Im Alltagsbetrieb kann beispielsweise im einen Schacht A5-Papier für Rezepte und AUF-Zeugnisse etc. eingelegt werden, im anderen A4-Papier für sonstige Dokumente. Beim Rechnungsdruck kann im einen Schacht Papier mit ESR-Schein und im anderen Normalpapier eingelegt werden.

\bigskip

Zusätzlich wird mindestens ein Labeldrucker erforderlich sein, um Etiketten für Laboraufträge etc. zu drucken. Hier sind die Anforderungen nicht weiter streng und hängen eher vom verwendeten Betriebssystem als von Elexis ab. Es können also auf einem Windows-PC auch USB-Labeldrucker verwendet werden, auf einem Linux-PC aber nur, wenn Treiber dafür existieren.

\subsection{USV}
Eine unterbrechungsfreie Stromversorgung sollte mindestens für den Server unbedingt empfohlen werden. Dazu gehört ein Programm, welches den Server kontrolliert herunterfährt, bevor die Stromreserve der USV erschöpft ist.

\subsection{Backup}
Ein funktionsfähig eingerichtetes Backup-System gehört zwingend zu jeder Elexis-Installation. Es ist Ihre Aufgabe, das Backup-System so einzurichten und vorzubereiten, dass der Kunde es einfach bedienen kann. Ob der Kunde es dann auch wirklich nutzt, ist hingegen natürlich nicht mehr Ihre Verantwortung. Auf das Backup gehen wir im Abschnitt \ref{backup} näher ein.

\subsection{Absicherung}
Falls das Netzwerk des Kunden am Internet angeschlossen ist, müssen Sie sich vergewissern, ob ein gutes Absicherungskonzept vorliegt, bzw. ggf. ein Solches für den Kunden einrichten. Die Vertraulichkeit der Daten, welche in der Elexis-Datenbank gespeichert sein werden, hat allerhöchste Priorität und Sie als Verantwortlicher für die Elexis-Installation und Konfiguration müssen darauf achten, dass das System des Kunden dieser Anforderung technisch genügt. (Ob und wie der Kunde die Sicherungsmassnahmen auch anwendet, liegt hingegen nicht mehr in Ihrer Verantwortung). Eine weitergehende Diskussion der Absicherungsmassnahmen ist im Abschnitt \ref{security}

\section{Vorbereitung der Installation}
\label{besprechung}
Es empfiehlt sich, den Kunden darauf aufmerksam zu machen, dass Sie folgende Angaben benötigen, damit er diese ggf. vorgängig sammeln und bereithalten kann:
\begin{itemize}
    \item ZSR und EAN-Nummer für jeden Mandanten
    \item Bankverbindung inklusive ESR-Teilnehmernummer und ESR-Kundennummer
    \item Eine konkrete Vorstellung, wie der Briefkopf und die verschiedenen Vorlagen für Arztbriefe, Rezepte, Rechnungen und Zeugnisse aussehen sollen.
    \item Eine Auflistung, wer alles am Programm arbeiten soll, mit Benutzernamen und Passwörtern.
    \item Eine Auflistung der im Praxislabor durchgeführten Untersuchungen, eingeteilt in logische Gruppen
    \item Auflistung der am Praxisstandort gültigen Taxpunktwerde für KVG, UVG, IV, MV.
    \item Aufstellung der am häufigsten benötigten Abrechnungspositionen und allfälliger Eigenleistungen und Eigenartikel.
\end{itemize}

Im weiteres sollte folgendes vor der Installation besorgt werden:
\begin{itemize}
    \item Jeweils neueste Version der Artikelliste (Medikamente, Medicals), MiGeL,  Analysenliste, Tarmed, ICD-10
    \item Liste der EAN aller Krankenkassen und UVG-Versicherer
    \item Installationspaket für den Datenbankserver
    \item Aktuellstes Installationspaket von Elexis

\end{itemize}

\section{Durchführung der Installation}
\label{ablauf}


\section{Grundkonfiguration}
\label{config}

\section{Instruktion des Kunden und dessen Personal}

\section{Zweitinstruktion}

\part{Datensicherungskonzepte}
\section{Backup}
\label{backup}

\section{Zugriffs-Sicherung}
\label{security}

\end{document}
