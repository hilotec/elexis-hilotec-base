%/*******************************************************************************
% * Copyright (c) 2007, G. Weirich
% * All rights reserved. This program may not be distributed
% * or modified without prior written consent
% *
% * Contributors:
% *    G. Weirich - initial implementation
% *
% *  $Id: anleitung.tex 291 2007-10-09 04:25:28Z Gerry $
% *******************************************************************************/

\documentclass[a4paper]{scrartcl}
\usepackage{german}
\usepackage[utf8]{inputenc}
\usepackage{makeidx}
\usepackage{wrapfig}
\makeindex
% Hier ein etwas skurriler Block, der dazu dient, die Unterschiede
% zwischen pdflatex und latex auszubügeln
% Grafiken müssen als png oder gif (für pdflatex) und als eps (für Latex)
% vorhanden sein. Die Endung kann man beim \includegraphics jeweils weglassen,
% das System nimmt je nach Renderer die geeignete Variante.

\newif\ifpdf
\ifx\pdfoutput\undefined
	\pdffalse              	%normales LaTeX wird ausgeführt
\else
	\pdfoutput=1
	\pdftrue               	%pdfLaTeX wird ausgeführt
\fi

\ifpdf
	\usepackage[pdftex]{graphicx}
	\DeclareGraphicsExtensions{.pdf,.jpg,.png}
\else
	\usepackage[dvips]{graphicx}
	\DeclareGraphicsExtensions{.eps}
\fi

\usepackage{floatflt}
\usepackage[]{hyperref}
\usepackage{color}
\title{Abrechnen mit Tarmed \& Co.}
\author{Gerry Weirich}

\begin{document}
\maketitle
\tableofcontents
\section{Einführung}
Der Weg vom Erbringen einer Leistung bis zum Einbuchen der Zahlung umfasst eine ganze Reihe verschiedener Schritte, die korrekt aufeinander abgestimmt werden müssen. Diese Broschüre erklärt die Konzepte und das Vorgehen beim Tarmed-System.

\medskip

Folgende Fragen müssen beantwortet werden, um korrekte Rechnungen zu erstellen und Zahlungen entgegenzunehmen\footnote{Dies ist nicht Elexis-spezifisch, sondern muss in jedem Praxisprogramm 'irgendwie' festgelegt werden. Bloss kann es bei weniger flexiblen Programmen 'festverdrahtet' sein, oder bei teuren Installationsveträgen kann die Konfiguration schon vom Hersteller vorgenommen werden sein. Da Elexis aber 'empowerment' des Anwenders auf den Fahnen stehen hat, müssen und dürfen Sie diese Dinge hier selber in die Hand nehmen (oder sich selber eine Supportfirma suchen, die es für Sie macht).}:

\begin{itemize}
\item Welches Tarifsystem und welche Tarifstufe wende ich an?
\item Für wen rechne ich ab?
\item An wen geht die Rechnung (und in welcher Form geht sie dorthin)?
\item Wer ist Kostenträger?
\item Wie identifiziere ich meine Leistungen gegenüber dem Kostenträger?
\item wohin sollen die Zahlungen gehen?
\item Woher weiss ich, ob Zahlungen eingegangen sind?
\item Wann und wie mahne ich?
\item Wann und wie leite ich Betreibungen ein?
\end{itemize}

Im Folgenden werde ich diese Schritte einzeln zeigen und die entsprechenden notwendigen Konfigurationen darlegen.

\section{Schritt für Schritt: Konfiguration}
Es sei an dieser Stelle nochmal darauf hingewiesen, dass diese Konfiguration nicht ganz trivial ist. Im Zweifelsfall sollten Sie es von einem Supporter durchführen lassen und stattdessen gleich im Abschnitt \ref{rechnungenerstellen} auf Seite \pageref{rechnungenerstellen} weiterlesen.

\medskip

\subsection{Abrechnungssysteme}
Die Leistungsverrechnung ist in Elexis als Plugin realisiert und damit auswechselbar oder auch ergänzbar - man kann nach Tarmed und nach CHOP, nach einem beliebigen Privatrechnungssystem und/oder nach einem österreichischen System abrechnen, das hängt nur davon ab, welche Plugins man installiert hat. Auch der Wechsel auf ein jetzt noch nicht bekanntes Tarifsystem ist damit kein Problem. Jedes Tarif-Plugin enthält das Codesystem selbst, sowie einen Mechanismus, um Rechnungen auszugeben, die zu diesem System konform sind (so muss eine Tarmed-Rechnung nicht nur andere Positionen beinhalten, sondern auch formal anders aussehen, als etwa eine Privatrechnung einer Heilpraktikerin). Ausserdem beinhaltet ein Abrechnungssystem je nachdem bestimmte Angaben, die vorhanden sein müssen, um Rechnungen erstellen zu können (z.B. Kostenträger, Versicherungsnummer u. Ä.)


\medskip

Leistungscode, Ausgabeziel und erforderliche Daten, sowie die Tarifstufe sind in Elexis als 'Abrechnungssystem' zusammengefasst (S. Abb. \ref{fig:abr1}):
\begin{figure}
  % Requires \usepackage{graphicx}
  \includegraphics[width=0.9\textwidth]{abr1}\\
  \caption{Abrechnungssysteme: Übersicht}\label{fig:abr1}
\end{figure}
Es kann beliebig viele Abrechnungssysteme geben, und die Namen der einzelnen Abrechnungssysteme sind seitens Elexis egal. Um alle Schweizer Systeme abzudecken, sollten allerdings Abrechnungssysteme mit den Namen 'KVG', 'UVG', 'IV', 'MV' und 'VVG' vorhanden sein. Weitere können nach Belieben dazukommen. Es ist beispielsweise problemlos möglich, zwei verschiedene KVG-Systeme mit verschiedenen Taxpunktwerten zu definieren.

\medskip

Um ein neues Abrechnungssystem zu erstellen, klickt man auf 'Neu...', um ein bestehendes zu modifizieren, klickt man doppelt darauf. Es öffnet sich der Abrechnungssystem-Konfigurationsdialog (Abb. \ref{fig:abr2}.
\begin{figure}
  % Requires \usepackage{graphicx}
  \includegraphics{abr2}\\
  \caption{KVG-Abrechnungssystem}\label{fig:abr2}
\end{figure}
Der Name ist wie gesagt frei wählbar. Für Leistungscode-System und Standard-Rechnungsausgabe können Sie aus den von den vorhandenen Abrechnungs-Plugins beigesteuerten Optionen wählen. Der Multiplikator ist die anzuwendende Tarifstufe (der 'Taxpunkt') für dieses System. Ein Multiplikator muss immer in Form einer Dezimalzahl angegeben werden und gilt immer ab einem bestimmten Datum so lange, bis ein anderer Multiplikator definiert wird. \textbf{Ein einmal eingegebener Multiplikator kann weder gelöscht noch geändert werden!}

Unter 'Notwendige Daten' geben Sie an, was für Angaben vorhanden sein müssen, damit ein Fall, der dieses Abrechnungssystem hat, als gültiog betrachtet wird. Solche Angaben können ein Kontakt sein, wie z.B. Kostenträger, oder ein Text, wie z.B. Versicherungs- oder Unfallnummer oder Vertragsnummer etc., oder es kann ein Datum sein, z.B. ein Unfalldatum.
Implizit immer notwendig ist der Kontakt 'Rechnungsempfänger', der deshalb hier nicht separat eingetragen werden darf.
Durch Rechtsklick kann man einen Eintrag in dieser Liste wieder löschen. Sie sehen das, was Sie hier eingegeben haben wieder unter den 'Fall-Details' als Fall-Erfordernisse (s. \ref{fall-erfordernisse}, S. \pageref{fall-erfordernisse}).

\subsection{Mandanten und Rechnungssteller}

Jeder Kontakt zwischen der Praxis und einem Patienten steht unter der Verantwortung eines \textit{Mandanten} und geht auf Rechnung eines \textit{Rechnungsstellers}. Im einfachsten und in der Schweiz üblichen Fall sind Mandant und Rechnungssteller identisch. Es ist aber auch möglich, dass ein Mandant im Fixlohn angestellt ist, auf eigen Verantwortung arbeitet, aber für einen anderen Rechnungssteller abrechnet (z.B. in einem HMO-Zentrum oder einer Polikinik). Dies ist nicht zu verwechseln mit einem Assistenten: Ein Assistent ist selber kein Mandant, sondern arbeitet auf Rechnung und unter der Verantwortung eines Mandanten.

\medskip

Einen Mandanten erstellt man, indem man ihn unter den \textit{Kontakten} anlegt und das Häkchen 'Mandant' ankreuzt. Danach kann man unter \textsc{Datei-Einstellungen-Mandanten} Benutzername und Passwort eingeben und festlegen, für welchen REchnungssteller dieser Mandant arbeitet.

Diese Zusammenhänge sind auch im Elexis-Handbuch beschrieben, genaueres bitte ich dort nachzulesen.

\medskip

Zurück zum Tarmed-System: Wenn das Plugin 'elexis-arzttarife-schweiz' installiert ist, finden Sie unter \textsc{Datei-Einstellungen-Abrechnungssysteme} auch die Positionen 'Labortarif' und 'Tarmed-Rechnungen' (S. Abb. \ref{fig:abr1}). Wählen Sie zunächst den Punkt Labortarife (Abb. \ref{fig:abr3}):
\begin{figure}
  % Requires \usepackage{graphicx}
  \includegraphics{abr3}\\
  \caption{Labor-Taxpunkt}\label{fig:abr3}
\end{figure}
Hier muss der Multiplikator gleich wie bei den Abrechnungssystemen eingegeben werden. Auch hier kann ein einmal eingetragener Wert nicht mehr geändert werden und er ist ab dem Stichdatum bis zur nächsten Änderung gültig.

\medskip

Gehen Sie dann zum Punkt 'Tarmed-Rechnungen'. Es erscheint ein Dialog wie in Abb. \ref{fig:abr4}.
\begin{figure}
  % Requires \usepackage{graphicx}
  \includegraphics[width=0.9\textwidth]{abr4}\\
  \caption{Tarmedrechnungs-Einstellungen}\label{fig:abr4}
\end{figure}

Hier müssen Sie für jeden Mandanten und Rechnungssteller \textit{separat} alle für die Abrechnung relevanten Daten eingeben. Wählen Sie also im Combobox.Feld ganz oben zunächst einen Mandanten aus. Dadurch erscheint im Feld hinter 'Leistungserbringer' der Name dieses Mandanten. Klicken Sie dann auf das blaue Wort 'Leistungserbringer'. Es öffnet sich eine Dialogbox wie in Abb \ref{fig:abr5}.
\begin{figure}
  % Requires \usepackage{graphicx}
  \includegraphics{abr5}\\
  \caption{Einstellung pro Mandant}\label{fig:abr5}
\end{figure}

Um eine Zeile zu ändern müssen Sie in dieser Zeile die \textbf{Eingabetaste} drücken, dann den neuen Wert eingeben, dann nochmal die \textbf{Eingabetaste} drücken und erst dann mit der Maus oder der Pfeiltaste ein anderes Feld aufsuchen.

Die Bedeutung der einzelnen Zeilen ist:
\begin{itemize}
\item Anrede: Sie haben es erraten ;-)
\item Kanton: Der Standort Ihrer Praxis. Wenn Sie in mehreren Kantonen praktizieren, müssen Sie einen eigenen Mandanten für jeden Kanton erstellen (z.B. Müller TI und Müller AG).
\item EAN: Ihre europäische Artikelnummer (Ja, wenn Sie ein in der Schweiz zugelassener Arzt sind, dann sind Sie auch ein europäischer Artikel, vergleichbar einer Milchtüte oder einem Badeschwamm). Falls Sie Ihre EAN nicht kennen, wenden Sie sich an die EAN.
\item NIF: Wenn Sie IV-Fälle behandeln, benötigen Sie eine Nummer der IV, die sich NIF nennt. Fragen Sie mich nicht, wieso die IV nicht die EAN verwenden kann.
\item KSK: Santésuisse hat Ihnen eine ZSR- oder KSK-Nummer gegeben, damit Sie über die Kassen abrechnen dürfen. Geben Sie diese Nummer ohne Punkte, Striche oder Leerzeichen ein. Und fragen Sie mich nicht, wieso Santésuisse nicht die EAN oder die NIF verwendet.
\item TarmedESR5OrEsr9: Das zu verwendende ESR-System. Fragen Sie  nicht, sondern schreiben Sie 'esr9'. Ausser, wenn Sie genau wissen, was Sie tun. Aber dann brauchen Sie eh nicht zu fragen.
\item TarmedEsrPlus: Dito. Schreiben Sie im Zweifelsfall einfach 'esr16or27'.
\item TarmedTiersGarantOrPayant: Schreiben Sie garant oder payant je nach Ihrem bevorzugten Abrechnungssystem. Hat aber zur Zeit nicht viel Konsequenz, was Sie da hinschreiben.
\item TarmedSpezialität: Der Titel, der Ihre 'Dignität' definiert. Fragen zum  Dignitätskonzept beantworten gerne und erschöpfend TarmedSuisse und die FMH.
\item TarmedKanton Nochmal der Praxisstandort.
\item TarmedErbringungsOrt praxis oder klinik (Wo Sie Ihre Leistungen erbringen)
\item TarmedDiagnoseSystem: Nach welcher Systematik Sie Standardmässig Diagnosen auf den Rechnungen angeben. Schreiben Sie TI für Tessiner-Code, ausser wenn Sie einen guten und von TarmedSuisse abgesegneten Grund haben, etwas anderes zu schreiben.
\end{itemize}

Klicken Sie dann auf 'OK' um diesen Dialog wieder zu schliessen und wieder zu Abb. \ref{fig:abr4} zurückzukommen.


\subsection{Post- oder Bankkonto}
Jetzt müssen Sie das Konto angeben, auf das Sie die Zahlungen für Ihre Rechnungen erwarten. Sie benötigen ein VESR oder BESR Konto, zusammenfassend auch ESR-Konto genannt. Ein VESR-Konto ist ein ESR-Konto bei der Post, ein BESR-Konto ist eines bei einer Bank. Ein ESR-Konto ist eines, das mit diesen orangen (früher blauen) Einzahlungsscheinen befüllt wird, auf denen eine Referenznummer steht und deren Mitteilungskästchen mit einem vorgedruckten 'Keine Mitteilungen anbringen' beschriftet ist.

Das schöne an ESR-Konti ist, dass Dateien mit Zahlungseingängen automatisch eingelesen und verarbeitet werden können, so dass Sie die Zahlungen nicht manuell verbuchen müssen, sondern diese Aufgabe an den Computer delegieren können.

\medskip

 \textbf{Dringende Empfehlung} Falls Sie schon ein ESR-Konto haben, das Sie mit einem anderen Programm benutzen, eröffnen Sie ein neues für die Benutzung mit Elexis, weil sonst die ESR-Zeilen nicht bzw. falsch interpretiert werden und beide Programme bei jedem Einlesen Fehlermeldungen ausspucken werden!

\medskip

Eröffnen Sie also je nach Geschmack bei der Post oder Ihrer Lieblingsbank ein ESR-Konto. Klicken Sie je nachdem auf das blaue Wort 'Postkonto' oder 'Bankverbindung'\footnote{Im Fall von 'Bankverbindung' muss Ihre Bank zuvor als Kontakt erfasst worden sein}.

\begin{figure}[htbp]
     \begin{minipage}{0.5\textwidth}
      \centering
       \includegraphics[width=0.95\textwidth]{abr6}
       \caption{VESR (Post)}
       	\label{fig:abr6}
     \end{minipage}\hfill
     \begin{minipage}{0.5\textwidth}
      \centering
       \includegraphics[width=0.95\textwidth]{abr7}
       \caption{BESR (Bank)}
       \label{fig:abr7}
     \end{minipage}
   \end{figure}

Je nach Auswahl erscheint eine Dialogbox wie Abb. \ref{fig:abr6} oder Abb. \ref{fig:abr7}. Beim Postkonto müssen Sie nur die VESR-Teilnehmernummer eintragen, die Sie von der Post erhalten haben. Beim Bankkonto benötigen Sie zunächst die Bank: Klicken Sie auf das blaue 'Finanzinstitut' und wählen Sie den hoffentlich zuvor angelegten Kontakt, der Ihre Bank beschreibt. Geben Sie dann neben TarmedESRParticipantNumber die VESR-Teilnehmernummer \textit{der Bank} ein. Neben TarmedESRIdentity müssen Sie die BESR-Identitätsnummer eingeben, die Sie von der Bank bekommen haben. Klicken Sie dann auf OK, um wieder zu Abb. \ref{fig:abr4} zurückzukommen.

\subsection{TrustCenter}
Da gibt es nicht viel zu sagen. Falls Sie einen TC-Vertrag haben, setzen Sie hier ein Häkchen und wählen Ihr TrustCenter aus. (Wenn Sie das tun, dann wird auf dem Rückforderungsbeleg von TG-Rechnungen ein passendes Token gedruckt, das beinahe wie eine ESR-Zeile aussieht, aber keine ist).

\subsection{Einzahlungsschein}
Elexis benötigt a4-Papier mit \textit{unten} integriertem ESR-Einzahlungsschein. Also das unterste Drittel der A4-Seite ist der orange Einzahlungsschein des Typs BESR oder VESR (bei ersterem steht eine Zeile 'zugunsten von' im Empfänger-Feld).

Zunächst benötigen Sie eine Schriftart des Typs OCR-B. diese Schriftart muss auf dem Computer installiert sein, von dem aus die Rechnungen gedruckt werden sollen \footnote{Tatsächlich empfehlen wir ausdrücklich, Rechnungen immer vom selben PC auf denselben Drucker auszugeben. Erst wenn das reibungslos funktioniert, können Experimente mit unterschiedlichen PC's gemacht werden, wenn es unbedingt sein muss.}. Nein, diese Schriftart ist nicht Bestandteil von Elexis, da sie lizenzpflichtig ist. Möglicherweise haben Sie aber von einem anderen ESR-fähigen Programm eine solche Schriftart irgendwo installiert und können diese verwenden. Ansonsten können Sie sich auch an die Post oder Ihre Bank wenden um zu erfahren, wo Sie diese Schriftart kaufen können. Sie benötigen Ausserdem eine ESR-Schablone, auf der Sie die korrekte Grösse und Positionierung der Codierzeile ablesen können.

\medskip

Nun zu den Einstellungen:
\begin{itemize}
\item Unter 'Schrift' geben Sie diejenige Schriftart ein, die der Einzahlungsschein ausserhalb der ESR-Zeile haben soll. Nehmen Sie hier irgendetwas nach Ihrem Geschmack (es muss natürlich eine auf dem PC installierte Schriftart sein).
\item Unter 'Grösse' geben Sie die gewünschte Schriftgrösse in Punkt ein.
\item Unter 'Schrift (Codierzeile)' müssen Sie die OCR-B Schriftart angeben.
\item Mit 'Grösse (Codierzeile)' müssen Sie etwas experimentieren, bis Sie die korrekte Grösse haben
\end{itemize}



Weil unterschiedliche Drucker das Papier nicht hunterprozentig identisch positionieren, und weil die Post ein wenig kritisch bezüglich der Position der ESR-Zeile ist, kann man hier den

\section{Schritt für Schritt: Verrechnen, Rechnungen erstellen und ausgeben}
\label{rechnungenerstellen}
\label{fall-erfordernisse}
\end{document}
